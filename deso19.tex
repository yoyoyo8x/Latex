\onehalfspacing
\section{Đề số 19}

\begin{bt} 
    \hfil
    \begin{enumerate}[a.]
        \item Tính giá trị biểu thức $A=\left(2 \frac{1}{3}+3,5\right):\left(-4 \frac{1}{6}+3 \frac{1}{7}\right)+7,5$
        \item Rút gọn biểu thức: $\quad B=\frac{2 \cdot 8^4 \cdot 27^2+4 \cdot 6^9}{2^7 \cdot 6^7+2^7 \cdot 40 \cdot 9^4}$
        \item Tìm đa thức $M$ biết rằng: $M+\left(5 x^2-2 x y\right)=6 x^2+9 x y-y^2$.
        Tính giá trị của M khi $x$, $y$ thỏa mãn $(2 x-5)^{2012}+(3 y+4)^{2014} \leq 0$.
    \end{enumerate}
\loigiai{}
\end{bt}

\begin{bt}
    \hfill
	\begin{enumerate}[a.]
        \item $\operatorname{Tim} x: \frac{1}{2}-\left|x+\frac{1}{5}\right|=\frac{1}{3}$
        \item Tìm $\mathrm{x}, \mathrm{y}, \mathrm{z}$ biết: $2 x=3 y ; 4 y=5 z$ và $x+y+z=11$
        \item Tìm $x$, biết : $(x+2)^{n+1}=(x+2)^{n+11}$ (Với $\mathrm{n}$ là số tự nhiên)
    \end{enumerate}
	\loigiai{} 
\end{bt}

\begin{bt}
    \hfill
	\begin{enumerate}[a.]
        \item Tìm độ dài 3 cạnh của tam giác có chu vi bằng $13 \mathrm{~cm}$. Biết độ dài 3 đường cao tương ứng lân lượt là $2 \mathrm{~cm}, 3 \mathrm{~cm}, 4 \mathrm{~cm}$.
        \item Tìm $x, y$ nguyên biết: $2 x y-x-y=2$
    \end{enumerate}
	\loigiai{}
\end{bt}

\begin{bt}
    Cho tam giác $\mathrm{ABC}\left(\mathrm{AB}<\mathrm{AC}\right.$, góc $\left.\mathrm{B}=60^{\circ}\right)$. Hai phân giác $\mathrm{AD}$ và $\mathrm{CE}$ của $\triangle \mathrm{ABC}$ cắt nhau ở $\mathrm{I}$, từ trung điểm $\mathrm{M}$ của $\mathrm{BC}$ kẻ đường vuông góc với đường phân giác $\mathrm{AI}$ tại $\mathrm{H}$, cắt $\mathrm{AB}$ ở $\mathrm{P}$, cắt $\mathrm{AC}$ ở $\mathrm{K}$.
    \begin{enumerate}[a.]
        \item  Tính AIC
        \item Tính độ dài cạnh $\mathrm{AK}$ biết $P K=6 \mathrm{~cm}, A H=4 \mathrm{~cm}$.
        \item Chứng minh $\Delta$ IDE cân.
    \end{enumerate}
\loigiai{}
\end{bt}

\begin{bt}
    Chứng minh rằng $\sqrt{10}$ là số vô tỉ.
\loigiai{}
\end{bt}

