\section{Đề số 44}

\begin{bt} 
	Tìm x biết:
	\begin{enumerate}[a.]
		\item $\frac{64}{(-2)^x}=(-16)^2: 4^3$
		\item $\frac{6}{x^2+2}+\frac{12}{x^2+8}=3-\frac{7}{x^2+3}$
		\item $|x-2|+|3-x|=11$
	\end{enumerate}
	\loigiai{} 
\end{bt}

\begin{bt}
	\hfill
	\begin{enumerate}[1.]
		\item Cho tỉ lệ thức $\frac{a}{b}=\frac{c}{d}$. Chứng minh rằng ta có các tỉ lệ thức sau ( giả thiết các tỉ lệ thức đều có nghĩa).
		\begin{enumerate}[a.]
			\item $\frac{4 a-3 b}{a}=\frac{4 c-3 d}{c}$
			\item $\frac{(a-b)^2}{(c-d)^2}=\frac{3 a^2+2 b^2}{3 c^2+2 d^2}$
		\end{enumerate}
		\item Tìm $x, y \in Z$ biết: $\mathrm{x}+\mathrm{y}+2 \mathrm{xy}=83$
	\end{enumerate}
	\loigiai{} 
\end{bt}

\begin{bt}
	\hfill
	\begin{enumerate}[a.]
		\item Hai xe máy cùng khởi hành 1 lúc từ $A$ và $B$ cách nhau $11 \mathrm{~km}$ để đi đến $C$ ( 3 địa điểm $A, B, C$ cùng ở trên một đường thẳng ) vận tốc của người đi từ $\mathrm{A}$ là $20 \mathrm{~km} / \mathrm{h}$, của người đi từ $\mathrm{B}$ là $24 \mathrm{~km} / \mathrm{h}$. Tính quãng đường mỗi người đã đi biết họ đến $\mathrm{C}$ cùng 1 lúc.
		\item Cho $\mathrm{f}(\mathrm{x})=a x^2+b x+c$ với $a, b, c \in Q$. Chứng tỏ rằng: $\mathrm{f}(-2)$. $\mathrm{f}(3) \leq 0$ biết $13 \mathrm{a}+\mathrm{b}+2 \mathrm{c}$ $=0$
	\end{enumerate}

	\loigiai{} 
\end{bt}

\begin{bt}
	Cho $\triangle \mathrm{ABC}$ có góc $\mathrm{B}$ và góc $\mathrm{C}$ là hai góc nhọn. Trên tia đối của tia $\mathrm{AB}$ lấy điểm $\mathrm{D}$ sao cho $\mathrm{AD}=\mathrm{AB}$, trên tia đối của tia $\mathrm{AC}$ lấy điểm $\mathrm{E}$ sao cho $\mathrm{AE}=\mathrm{AC}$.
	\begin{enumerate}[a.]
		\item Chứng minh rằng: $\mathrm{BE}=\mathrm{CD}$
		\item Lấy $M$ là trung điểm của $\mathrm{BE}, \mathrm{N}$ là trung điểm của $C D$. Chứng minh $\mathrm{M}, \mathrm{A}, \mathrm{N}$ thẳng hàng.
		\item $\mathrm{Ax}$ là tia bất kì nằm giữa 2 tia $\mathrm{AB}$ và $\mathrm{AC}$. Gọi $\mathrm{H}, \mathrm{K}$ lân lượt là hình chiếu của $\mathrm{B}$ và $\mathrm{C}$ trên tia $\mathrm{Ax}$. Chứng minh $B H+C K \leq B C$
		\item Xác định vị trí của tia $\mathrm{Ax}$ để tổng $\mathrm{BH}+\mathrm{CK}$ có giá trị lớn nhất.
	\end{enumerate}
	\loigiai{}
\end{bt}

\begin{bt}
	Cho biểu thức $\mathrm{A}=\frac{3|x|+2}{4|x|-5}$ \\Tìm $x \in Z$ để $\mathrm{A}$ đại GTLN, tìm GTLN đó.
	\loigiai{}
\end{bt}
