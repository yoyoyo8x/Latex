\section{Đề số 41}

\begin{bt} 
	\hfill
	\begin{enumerate}[a.]
		\item Tính giá trị biểu thức: $\mathrm{A}=\frac{2^{12} \cdot 13+2^{12} \cdot 65}{2^{10} \cdot 104}+\frac{3^{10} \cdot 11+3^{10} \cdot 5}{3^9 \cdot 2^4}$
		\item Cho $\mathrm{A}=3+3^2+3^3+\ldots+3^{2015}$. Tìm số tự nhiên $\mathrm{n}$ biết rằng $2 \mathrm{~A}+3=3^{\mathrm{n}}$
	\end{enumerate}
	\loigiai{} 
\end{bt}

\begin{bt}
	\hfill
	\begin{enumerate}[a.]
		\item Tìm các số $\mathrm{x} ; \mathrm{y} ; \mathrm{z}$ biết rằng: $\frac{y+z+1}{x}=\frac{x+z+2}{y}=\frac{y+x-3}{z}=\frac{1}{x+y+z}$
		\item Tìm $x: \frac{x+4}{2012}+\frac{x+3}{2013}=\frac{x+2}{2014}+\frac{x+1}{2015}$
		\item Tìm $x$ để biểu thức sau nhận giá trị dương: $x^2+2016 x$
	\end{enumerate}
	\loigiai{} 
\end{bt}

\begin{bt}
	\hfill 
	\begin{enumerate}[a.]
		\item Cho $A=\frac{\sqrt{x}+1}{\sqrt{x}-3}$. Tìm số nguyên $\mathrm{x}$ để $\mathrm{A}$ là số nguyên
		\item Tìm giá trị lớn nhất của biểu thức: $\mathrm{B}=\frac{x^2+15}{x^2+3}$
		\item Tìm số nguyên $x, y$ sao cho $x-2 x y+y=0$
	\end{enumerate}
	\loigiai{} 
\end{bt}

\begin{bt}
	Cho tam giác $A B C, M$ là trung điểm của $B C$. Trên tia đối của của tia MA lấy điểm $E$ sao cho $\mathrm{ME}=\mathrm{MA}$. Chứng minh rằng:
	\begin{enumerate}[a.]
		\item $A C=E B$ và $A C / / B E$
		\item Gọi $\mathrm{I}$ là một điểm trên $\mathrm{AC}$; $\mathrm{K}$ là một điểm trên $\mathrm{EB}$ sao cho $\mathrm{AI}=\mathrm{EK}$. Chứng minh ba điểm $\mathrm{I}, \mathrm{M}, \mathrm{K}$ thẳng hàng
		\item Từ E kẻ $E H \perp B C(H \in B C)$. Biết $H B E=50^{\circ} ; M E B=25^{\circ}$.
		Tính $H E M$ và $B M E$
	\end{enumerate}
	\loigiai{}
\end{bt}

\begin{bt}
	Từ điểm I tùy ý trong tam giác $\mathrm{ABC}$, kẻ $\mathrm{IM}$, IN, IP lân lượt vuông góc với $\mathrm{BC}, \mathrm{CA}$, $\mathrm{AB}$. Chứng minh rằng: $\mathrm{AN}^2+\mathrm{BP}^2+\mathrm{CM}^2=\mathrm{AP}^2+\mathrm{BM}^2+\mathrm{CN}^2$
	\loigiai{}
\end{bt}
