\onehalfspacing
\section{Đề số 18}
\graphicspath{{./img/}}
\begin{bt} 
    \hfil
    \begin{enumerate}[a.]
        \item So sánh: $\sqrt{17}+\sqrt{26}+1$ và $\sqrt{99}$.
        \item Chứng minh: $\frac{1}{\sqrt{1}}+\frac{1}{\sqrt{2}}+\frac{1}{\sqrt{3}}+\ldots .+\frac{1}{\sqrt{99}}+\frac{1}{\sqrt{100}}>10$.
        \item Cho $S=1-\frac{1}{2}+\frac{1}{3}-\frac{1}{4}+\ldots+\frac{1}{2013}-\frac{1}{2014}+\frac{1}{2015}$ và
        $$
        P=\frac{1}{1008}+\frac{1}{1009}+\frac{1}{1010}+\ldots+\frac{1}{2014}+\frac{1}{2015} \text {. }
        $$
        Tính $(S-P)^{2016}$.
    \end{enumerate}
\loigiai{}
\end{bt}

\begin{bt}
    \hfill
	\begin{enumerate}[a.]
        \item Một số nguyên tố $\mathrm{p}$ chia cho 42 có số dư $\mathrm{r}$ là hợp số. Tìm hợp số $\mathrm{r}$.
        \item Tìm số tự nhiên $\overline{a b}$ sao cho $\overline{a b}^2=(a+b)^3$
    \end{enumerate}
	\loigiai{} 
\end{bt}

\begin{bt}
    \hfill
	\begin{enumerate}[a.]
        \item Cho $\mathrm{x} ; \mathrm{y} ; \mathrm{z} \neq 0$ và $\mathrm{x}-\mathrm{y}-\mathrm{z}=0$. Tính giá trị biểu thức $B=\left(1-\frac{z}{x}\right)\left(1-\frac{x}{y}\right)\left(1+\frac{y}{z}\right)$
        \item Cho $\frac{3 x-2 y}{4}=\frac{2 z-4 x}{3}=\frac{4 y-3 z}{2}$. Chứng minh rằng: $\frac{x}{2}=\frac{y}{3}=\frac{z}{4}$
         Cho biểu thức $M=\frac{5-x}{x-2}$. Tìm x nguyên để $\mathrm{M}$ có giá trị nhỏ nhất.
    \end{enumerate}
	\loigiai{}
\end{bt}

\begin{bt}
   Cho $x A y=60^{\circ}$ vẽ tia phân giác $\mathrm{Az}$ của góc đó. Từ một điểm $\mathrm{B}$ trên tia $\mathrm{Ax}$ vẽ đường thẳng song song với $\mathrm{Ay}$ cắt $\mathrm{Az}$ tại $\mathrm{C}$. Kẻ $\mathrm{BH} \perp \mathrm{Ay}$ tại $\mathrm{H}, \mathrm{CM} \perp \mathrm{Ay}$ tại $\mathrm{M}, \mathrm{BK} \perp$ AC tại K. Chứng minh:
    \begin{enumerate}[a.]
        \item $\mathrm{KC}=\mathrm{KA}$
        \item $\mathrm{BH}=\frac{A C}{2}$
        \item $\triangle \mathrm{KMC}$ đều.
    \end{enumerate}
\loigiai{}
\end{bt}

\begin{bt}
   Cho $\Delta \mathrm{ABC}$ có $B=2 \cdot C<90^{\circ}$. Vẽ $\mathrm{AH}$ vuông góc với $\mathrm{BC}$ tại $\mathrm{H}$. Trên tia $\mathrm{AB}$ lấy điểm $\mathrm{D}$ sao cho $\mathrm{AD}=\mathrm{HC}$. Chứng minh rằng đường thẳng $\mathrm{DH}$ đi qua trung điểm của đoạn thẳng $\mathrm{AC}$.
\loigiai{}
\end{bt}

