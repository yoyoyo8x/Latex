\onehalfspacing
\section{Đề số 5}

\begin{bt} 
	\hfill
	\begin{enumerate}[a.]
		\item Tìm $x, y$ biết: $\frac{4+\mathrm{x}}{7+\mathrm{y}}=\frac{4}{7}$ và $\mathrm{x}+\mathrm{y}=22$
		\item Cho $\frac{x}{3}=\frac{y}{4}$ và $\frac{y}{5}=\frac{z}{6}$. Tính $M=\frac{2 x+3 y+4 z}{3 x+4 y+5 z}$
	\end{enumerate}
	\loigiai{} 
\end{bt}

\begin{bt}
	Thực hiện tính:
	\begin{enumerate}[a.]
		\item $S=2^{2010}-2^{2009}-2^{2008} \ldots-2-1$
		\item $\mathrm{P}=1+\frac{1}{2}(1+2)+\frac{1}{3}(1+2+3)+\frac{1}{4}(1+2+3+4)+\ldots+\frac{1}{16}(1+2+3+\ldots+16)$
	\end{enumerate}
	\loigiai{} 
\end{bt}

\begin{bt}
	Tìm x biết:
	\begin{enumerate}[a.]
		\item 
		 $\frac{1}{4} \cdot \frac{2}{6} \cdot \frac{3}{8} \cdot \frac{4}{10} \cdot \frac{5}{12} \ldots \frac{30}{62} \cdot \frac{31}{64}=2^{\mathrm{x}}$
		\item $\frac{4^5+4^5+4^5+4^5}{3^5+3^5+3^5} \cdot \frac{6^5+6^5+6^5+6^5+6^5+6^5}{2^5+2^5}=2^x$
	\end{enumerate}
	\loigiai{} 
\end{bt}

\begin{bt}
	Cho tam giác $A B C$ có $B<90^{\circ}$ và $B=2 C$. Kẻ đường cao $A H$. Trên tia đối của tia $B A$ lấy điểm $\mathrm{E}$ sao cho $\mathrm{BE}=\mathrm{BH}$. Đường thẳng $\mathrm{HE}$ cắt $\mathrm{AC}$ tại $\mathrm{D}$. 
	\begin{enumerate}[a.]
		\item Chứng minh $\mathrm{BEH}=\mathrm{ACB}$
		\item Chứng $\operatorname{minh} \mathrm{DH}=\mathrm{DC}=\mathrm{DA}$.
		\item Lấy $\mathrm{B}^{\prime}$ sao cho $\mathrm{H}$ là trung điểm của $\mathrm{BB}^{\prime}$. Chứng minh tam giác $\mathrm{AB}^{\prime} \mathrm{C}$ cân.
		\item Chứng minh $\mathrm{AE}=\mathrm{HC}$.
	\end{enumerate}
	\loigiai{}
\end{bt}