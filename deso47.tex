\section{Đề số 47}

\begin{bt} 
	\hfill
	\begin{enumerate}[a.]
		\item Thực hiện phép tính: $\mathrm{A}=\frac{2^{12} \cdot 3^5-4^6 \cdot 9^2}{\left(2^2 \cdot 3\right)^6+8^4 \cdot 3^5}-\frac{5^{10} \cdot 7^3-25^2 \cdot 49^2}{(125 \cdot 7)^3+5^9 \cdot 14^3}$
		\item Chứng minh rằng : $\frac{1}{7^2}-\frac{1}{7^4}+\ldots+\frac{1}{7^{4 n-2}}-\frac{1}{7^{4 n}}+\ldots+\frac{1}{7^{98}}-\frac{1}{7^{100}}<\frac{1}{50}$
		\item Tính: $B=1^2+2^2+3^2+4^2+5^2+\ldots \ldots \ldots+98^2$
		\item Cho $\mathrm{p}$ là số nguyên tố lớn hon 3 chứng minh rằng: $\mathrm{p}^2-1$ chia hết cho 24
	\end{enumerate}
	\loigiai{} 
\end{bt}

\begin{bt}
	\hfill
	\begin{enumerate}[a.]
		\item Tìm $x$ biết $\left|x-\frac{1}{3}\right|+\frac{4}{5}=\left|(-3,2)+\frac{2}{5}\right|$
		\item Cho $\mathrm{C}=\frac{m^3+3 m^2+2 m+5}{m(m+1)(m+2)+6}$ với $\mathrm{m} \in N$ Chứng minh $\mathrm{C}$ là số hữu tỉ
		\item Cho $M=(x-1)(x+2)(3-x)$. Tìm $x$ để $M<0$
	\end{enumerate}
	\loigiai{} 
\end{bt}

\begin{bt}
	\hfill
	\begin{enumerate}[a.]
		\item Cho $\frac{a}{c}=\frac{c}{b}$ chứng minh rằng: $\frac{a^2+c^2}{b^2+c^2}=\frac{a}{b}$
		\item Tìm các giá trị nguyên của $x$ và $y$ biết: $x^2-y^2=5$
	\end{enumerate}
	
	\loigiai{} 
\end{bt}

\begin{bt}
	Cho tam giác $A B C$ có $B A C=75^{\circ}, A B C=35^{\circ}$. Phân giác của góc $B A C$ cắt cạnh $B C$ tại $D$. Đường thẳng qua $A$ và vuông góc với $A D$ cắt tia $B C$ tai $E$. Gọi $M$ là trung điểm của $D E$. Chøng minh rằng:
	\begin{enumerate}[a.]
		\item Tam giác $A C M$ là tam giác cân.
		\item $A B<\frac{A D+A E}{2}$
		\item Chu vi tam giác $A B C$ bằng độ dài đoạn thẳng $B E$.
	\end{enumerate}
	\loigiai{}
\end{bt}

\begin{bt}
	\hfill
	\begin{enumerate}[a.]
		\item Tìm một số có 3 chữ số,biết rằng số đó chia hết cho 18 và các chữ số của nó tỉ lệ với 1,2 và 3 .
		\item Cho $f(x)=3 x^2-2 x-1$ Tìm $x$ để $f(x)=0$
	\end{enumerate}
	\loigiai{} 
\end{bt}

