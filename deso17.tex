\onehalfspacing
\section{Đề số 17}
\graphicspath{{./img/}}
\begin{bt} 
    Tính giá trị biểu thức:
$$
\mathrm{A}=\frac{(a+b)(-x-y)-(a-y)(b-x)}{a b x y(x y+a y+a b+b y)} \text { Với } a=\frac{1}{3} ; b=-2 ; x=\frac{3}{2} ; y=1
$$
	\loigiai{} 
\end{bt}

\begin{bt}
	Chứng minh rằng: Nếu $0<a_1<a_2<\ldots<a_9$ thì: $\frac{a_1+a_2+\ldots .+a_9}{a_3+a_6+a_9}<3$
	\loigiai{} 
\end{bt}

\begin{bt}
	Có 3 mảnh đất hình chữ nhật: $A$; $B$ và $C$. Các diện tích của $A$ và $B$ tỉ lệ với 4 và 5 , các diện tích của $B$ và $C$ tỉ lệ với 7 và $8 ; A$ và $B$ có cùng chiều dài và tổng các chiều rộng của chúng là $27 \mathrm{~m}$. B và $C$ có cùng chiều rộng. Chiều dài của mảnh đất $C$ là $24 \mathrm{~m}$. Hãy tính diện tích của mỗi mảnh đất đó.
	\loigiai{}
\end{bt}

\begin{bt}
    Cho 2 biểu thức:
$$
\mathrm{A}=\frac{4 x-7}{x-2} ; \quad \mathrm{B}=\frac{3 x^2-9 x+2}{x-3}
$$
    \begin{enumerate}[a.]
        \item Tìm giá trị nguyên của $x$ để mỗi biểu thức có giá trị nguyên
        \item Tìm giá trị nguyên của $x$ để cả hai biểu thức cùng có giá trị nguyên.
    \end{enumerate}
\loigiai{}
\end{bt}

\begin{bt}
    Cho tam giác cân $\mathrm{ABC}, \mathrm{AB}=\mathrm{AC}$. Trên tia đối của các tia $\mathrm{BC}$ và $\mathrm{CB}$ lấy theo thứ tự hai điểm $\mathrm{D}$ và $\mathrm{E}$ sao cho $\mathrm{BD}=\mathrm{CE}$
    \begin{enumerate}[a.]
        \item Chứng minh tam giác $\mathrm{ADE}$ là tam giác cân.
        \item Gọi $\mathrm{M}$ là trung điểm của $\mathrm{BC}$. Chứng minh $\mathrm{AM}$ là tia phân giác của góc $\mathrm{DAE}$
        \item Từ $\mathrm{B}$ và $\mathrm{C}$ vẽ $\mathrm{BH}$ và $\mathrm{CK}$ theo thứ tự vuông góc với $\mathrm{AD}$ và $\mathrm{AE}$. Chứng minh $\mathrm{BH}=\mathrm{CK}$
        \item Chứng minh 3 đường thẳng $\mathrm{AM} ; \mathrm{BH} ; \mathrm{CK}$ gặp nhau tại 1 điểm.
    \end{enumerate}
\loigiai{}
\end{bt}

