\onehalfspacing
\section{Đề số 11}
\graphicspath{{./img/}}
\begin{bt} 
    \hfill
	\begin{enumerate}[a.]
		\item Tính giá trị của biểu thức: $A=\frac{4}{9}:\left(\frac{1}{15}-\frac{2}{3}\right)+\frac{4}{9}:\left(\frac{1}{11}-\frac{5}{22}\right)$
        \item Tìm $x$, biết: $\left(-1 \frac{3}{5}+x\right): \frac{12}{13}=2 \frac{1}{6}$
        \item Tính giá trị của biểu thức $M=21 x^2 y+4 x y^2$ với $x$, $y$ thoả mãn:
        $$
        (x-2)^4+(2 y-1)^{2014} \leq 0
        $$
	\end{enumerate}
	\loigiai{} 
\end{bt}

\begin{bt}
	\hfill
	\begin{enumerate}[a.]
		\item Tìm các số $\mathrm{x}, \mathrm{y}, \mathrm{z}$ biết: $\frac{x}{3}=\frac{y}{4} ; \quad \frac{y}{6}=\frac{z}{8} \quad$ và $2 x+y-z=-14$.
        \item Tìm $x$, biết: $(x-2)\left(x+\frac{2}{3}\right)>0$.
        \item Tìm số nguyên $x$, biết rằng: $\frac{3}{7} \cdot 15 \frac{1}{3}+\frac{3}{7} \cdot 5 \frac{2}{5} \leq x \leq\left(3 \frac{1}{2}: 7-6 \frac{1}{2}\right) \cdot\left(-2 \frac{1}{3}\right)$
	\end{enumerate}
	\loigiai{} 
\end{bt}

\begin{bt}
	\hfill
	\begin{enumerate}[a.]
		\item Tính giá trị của biểu thức $\mathrm{M}=4 \mathrm{x}+4 \mathrm{y}+21 \mathrm{xy}(\mathrm{x}+\mathrm{y})+7\left(\mathrm{x}^3 \mathrm{y}^2+\mathrm{x}^2 \mathrm{y}^3\right)+2014$, biết $\mathrm{x}+\mathrm{y}=0$.
        \item Cho đa thức $p(x)=a x^3+b x^2+c x+d$, với $a, b, c, d$ là các hệ số nguyên. Biết rằng, $\mathrm{p}(\mathrm{x}) \vdots 5$ với mọi $\mathrm{x}$ nguyên. Chứng minh rằng $\mathrm{a}, \mathrm{b}, \mathrm{c}, \mathrm{d}$ đều chia hết cho 5 .
        \item Cho $A=1+\frac{1}{2}+\frac{1}{3}+\frac{1}{4}+\ldots+\frac{1}{4026}, B=1+\frac{1}{3}+\frac{1}{5}+\frac{1}{7}+\ldots+\frac{1}{4025}$. So sánh $\frac{A}{B}$ với $1 \frac{2013}{2014}$.
	\end{enumerate}
	\loigiai{} 
\end{bt}

\begin{bt}
	Cho tam giác $A B C$ cân tại $A$. Trên cạnh $B C$ lấy điểm $D$ ( $D$ khác $B, C$ ). Trên tia đối của tia $C B$, lấy điểm $\mathrm{E}$ sao cho $\mathrm{CE}=\mathrm{BD}$. Đường vuông góc với $\mathrm{BC}$ kẻ từ $\mathrm{D}$ cắt $\mathrm{BA}$ tại $\mathrm{M}$. Đường vuông góc với $\mathrm{BC}$ kẻ từ $\mathrm{E}$ cắt tia $\mathrm{AC}$ tại $\mathrm{N}$. $\mathrm{MN}$ cắt $\mathrm{BC}$ tại $\mathrm{I}$.
	\begin{enumerate}[a.]
		\item Chứng minh rằng: $\mathrm{DM}=\mathrm{EN}$.
        \item Chứng minh rằng $\mathrm{IM}=\mathrm{IN} ; \mathrm{BC}<\mathrm{MN}$.
        \item Gọi $\mathrm{O}$ là giao của đường phân giác góc $\mathrm{A}$ và đường thẳng vuông góc với $\mathrm{MN}$ tại $\mathrm{I}$. Chứng minh rằng: $\triangle B M O=\triangle C N O$. Từ đó suy ra điểm $\mathrm{O}$ cố định.
	\end{enumerate}
	\loigiai{}
\end{bt}

\begin{bt}
Cho tam giác $\mathrm{ABC}$ cân tại $\mathrm{A}$. Trên đường trung tuyến $\mathrm{BD}$ lấy điểm $\mathrm{E}$ sao cho $D A E=A B D$ (E nằm giữa $\mathrm{B}$ và $\mathrm{D}$ ). Chứng minh rằng $D A E=E C B$.
\loigiai{}
\end{bt}