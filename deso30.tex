\onehalfspacing
\section{Đề số 30}

\begin{bt} 
    Thực hiện phép tính:
    $$
    \mathrm{A}=\left(1-\frac{1}{2}\right)\left(1-\frac{1}{3}\right)\left(1-\frac{1}{4}\right) ; \quad \mathrm{B}=(0,25)^2 \cdot\left(\frac{1}{3}\right)^{-2}\left(\frac{4}{3}\right)^2 \cdot\left(\frac{1}{4}\right)^{-1}
    $$
\loigiai{}
\end{bt}

\begin{bt}
    \hfill
	\begin{enumerate}[a.]
        \item Tìm x biết: $|2 x-6|-4 x=12$
        \item Tìm $x$ biết: $\left(\frac{1}{2}+\frac{1}{3}+\ldots+\frac{1}{2015}\right) \cdot x=\frac{2014}{1}+\frac{2013}{2}+\ldots+\frac{2}{2013}+\frac{1}{2014}$
        \item Chứng minh rằng: Nếu $\frac{a}{b}=\frac{c}{d}$ thì $\frac{4 a+5 b}{4 a-5 b}=\frac{4 c+5 d}{4 c-5 d}$
        (Với $a, b, c, d \neq 0 ; 4 a \neq \pm 5 b ; 4 c \neq \pm 5 d$ )
    \end{enumerate}
	\loigiai{} 
\end{bt}

\begin{bt}
    Một vật chuyển động trên các cạnh hình vuông. Trên hai cạnh đầu vật chuyên động với vận tốc $5 \mathrm{~cm} / \mathrm{s}$, trên cạnh thứ ba với vận tốc $4 \mathrm{~cm} / \mathrm{s}$, trên cạnh thứ tư với vận tốc $3 \mathrm{~cm} / \mathrm{s}$. Hỏi độ dài cạnh hình vuông biết rằng tổng thời gian vật chuyển động trên bốn cạnh là 59 giây.
	\loigiai{}
\end{bt}

\begin{bt}
    Cho tam giác $A B C$ cân tại $A$. Trên cạnh $B C$ lấy điêm $D$, trên tia đối của tia $C B$ lấy điểm $\mathrm{E}$ sao cho $\mathrm{BD}=\mathrm{CE}$. Các đường thẳng vuông góc với $\mathrm{BC}$ kẻ từ $\mathrm{D}$ và $\mathrm{E}$ cắt $\mathrm{AB}, \mathrm{AC}$ lần lượt ở $M$, $N$.
    \begin{enumerate}[a.]
        \item  Chứng minh rằng: $\mathrm{DM}=\mathrm{EN}$.
        \item $\mathrm{MN}$ cắt $\mathrm{BC}$ tại $\mathrm{I}$.Chứng minh $\mathrm{I}$ là trung điểm của $\mathrm{MN}$.
        \item Chứng minh rằng đường thẳng vuông góc với $\mathrm{MN}$ tại $I$ luôn đi qua một điểm cố định khi $D$ thay đổi trên cạnh $BC$.
    \end{enumerate}
	\loigiai{}
\end{bt}

\begin{bt}
    Cho $f(x)=a x^2+b x+c$ với $a, b, c$ là các số hữu tỉ. Chứng tỏa rằng: $f(-2) \cdot f(3) \leq 0$.

    Biết rằng $13 a+b+2 c=0$
\loigiai{}
\end{bt}


