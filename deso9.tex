\onehalfspacing
\section{Đề số 9}

\begin{bt}
	\hfill
	\begin{enumerate}[1.]
		\item Tính giá trị các biểu thức sau:
		\begin{enumerate}[a.]
			\item $\mathrm{A}=\left(\frac{-3}{7}+\frac{4}{11}\right): \frac{7}{11}+\left(\frac{-4}{7}+\frac{7}{11}\right): \frac{7}{11}$
			\item $B=\frac{2^{12} \cdot 3^5-4^6 \cdot 9^2}{\left(2^2 \cdot 3\right)^6+8^4 \cdot 3^5}$
		\end{enumerate}
		\item Cho $\frac{x}{3}=\frac{y}{5}$. Tính giá trị biểu thức: $C=\frac{5 x^2+3 y^2}{10 x^2-3 y^2}$
	\end{enumerate}
	\loigiai{} 
\end{bt}

\begin{bt}
	\hfill
	\begin{enumerate}[1.]
		\item Tìm các số $x, y, z$, biết:
		\begin{enumerate}[a.]
			\item $\frac{x}{2}=\frac{y}{3} ; \frac{y}{5}=\frac{z}{7}$ và $x+y+z=92$
			\item $(x-1)^{2018}+(2 y-1)^{2018}+|x+2 y-z|^{2019}=0$
		\end{enumerate}
		\item Tìm $x, y$ nguyên biết: $x y+3 x-y=6$
	\end{enumerate}
	\loigiai{} 
\end{bt}

\begin{bt}
	\hfill
	\begin{enumerate}[1.] 
		\item Tìm đa thức $\mathrm{A}$ biết: $A-\left(3 x y-4 y^2\right)=x^2-7 x y+8 y^2$
		\item Cho hàm số $y=f(x)=a x+2$ có đồ thị đi qua điểm $A\left(a-1 ; a^2+a\right)$.
		\begin{enumerate}[a.]
			\item Tìm a
			\item Với a vừa tìm được, tìm giá trị của $x$ thỏa mãn: $f(2 x-1)=f(1-2 x)$
		\end{enumerate}
	\end{enumerate}
	\loigiai{} 
\end{bt}

\begin{bt}
	Cho tam giác $A B C$ vuông tại $A$. Vẽ về phía ngoài tam giác $A B C$ các tam giác đều $A B D$ và $A C E$. Gọi $\mathrm{I}$ là giao điểm $B E$ và $C D$. Chứng minh rằng:
	\begin{enumerate}[a.]
		\item $B E=C D$
		\item $\triangle \mathrm{BDE}$ là tam giác cân
		\item $\mathrm{EIC}=60^{\circ}$ và IA là tia phân giác của DIE
	\end{enumerate}
	\loigiai{}
\end{bt}

\begin{bt}
	\hfill
	\begin{enumerate}[1.] 
		\item Tìm số hữu tỉ $x$, sao cho tổng của số đó với nghịch đảo của nó có giá trị là một số nguyên.
		\item Cho các số a,b,c không âm thỏa mãn: $a+3 c=2016 ; a+2 b=2017$. Tìm giá trị lớn nhất của biểu thức $P=a+b+c$.
	\end{enumerate}
	\loigiai{}
\end{bt}