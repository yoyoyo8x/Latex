\section{Đề số 50}

\begin{bt} 
	\hfill
	\begin{enumerate}[1.]
		\item Rút gọn: $A=\left(\frac{3}{2}-\frac{2}{5}+\frac{1}{10}\right):\left(\frac{3}{2}-\frac{2}{3}+\frac{1}{12}\right)$.
		\item Tìm giá trị nhỏ nhất của biểu thức $P=|x-2012|+|x-2013|$ với $x$ là số tự nhiên.
	\end{enumerate}
	\loigiai{} 
\end{bt}

\begin{bt}
	\hfill
	\begin{enumerate}[1.]
		\item Tìm $x$ biết $2^{x+2} \cdot 3^{x+1} \cdot 5^x=10800$.
		\item $B a$ bạn $A n$, Bình và Cường có tổng số viên bi là 74. Biết rằng số viên bi của An và Bình tỉ lệ với 5 và 6 ; số viên bi của Bình và Cường tỉ lệ với 4 và 5 . Tính số viên bi của mỗi bạn.
	\end{enumerate}
		\loigiai{} 
\end{bt}
	
\begin{bt}
		\hfill
		\begin{enumerate}[1.]
			\item Cho $p$ là số nguyên tố lớn hơn 3 . Chứng minh rằng $p^2+2012$ là hợp số.
			\item Cho $n$ là số tự nhiên có hai chữ số. Tìm $n$ biết $n+4$ và $2 n$ đều là các số chính phương.
		\end{enumerate}
		
		\loigiai{} 
\end{bt}
\begin{bt}
	Cho tam giác $\mathrm{ABC}$ cân tại $\mathrm{A}$ và có cả ba góc đều là góc nhọn.
	\begin{enumerate}[1.]
		\item Về phía ngoài của tam giác vẽ tam giác $A B E$ vuông cân ở $B$. Gọi $\mathrm{H}$ là trung điểm của $\mathrm{BC}$, trên tia đối của tia $\mathrm{AH}$ lấy điểm $\mathrm{I}$ sao cho $A I=B C$. Chứng minh hai tam giác $A B I$ và $B E C$ bằng nhau và $B I \perp C E$.
		\item Phân giác của các góc $A B C, B D C$ cắt $\mathrm{AC}, \mathrm{BC}$ lần lượt tại $\mathrm{D}, \mathrm{M}$. Phân giác của góc $B D A$ cắt $\mathrm{BC}$ tại $\mathrm{N}$. Chứng minh rằng: $B D=\frac{1}{2} M N$.
	\end{enumerate}
	\loigiai{}
\end{bt}

\begin{bt}
	Cho $S=1-\frac{1}{2}+\frac{1}{3}-\frac{1}{4}+\ldots+\frac{1}{2011}-\frac{1}{2012}+\frac{1}{2013} \quad$ và $\quad P=\frac{1}{1007}+\frac{1}{1008}+\ldots+\frac{1}{2012}+\frac{1}{2013}$. Tính $(S-P)^{2013}$
	\loigiai{} 
\end{bt}
	
