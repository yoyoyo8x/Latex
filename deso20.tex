\onehalfspacing
\section{Đề số 20}
\graphicspath{{./img/}}
\begin{bt} 
    \hfil
    \begin{enumerate}[a.]
        \item Tính $M=\left(\frac{0,4-\frac{2}{9}+\frac{2}{11}}{1,4-\frac{7}{9}+\frac{7}{11}}-\frac{\frac{1}{3}-0,25+\frac{1}{5}}{1 \frac{1}{6}-0,875+0,7}\right): \frac{2017}{2018}$.
        \item Tìm x, biết: $|2017-x|+|2018-x|+|2019-x|=2$.
    \end{enumerate}
\loigiai{}
\end{bt}

\begin{bt}
    \hfill
	\begin{enumerate}[a.]
        \item Cho a, b, c là ba số thực dương thỏa mãn điều kiện:
        $$
        \frac{\mathrm{a}+\mathrm{b}-\mathrm{c}}{\mathrm{c}}=\frac{\mathrm{b}+\mathrm{c}-\mathrm{a}}{\mathrm{a}}=\frac{\mathrm{c}+\mathrm{a}-\mathrm{b}}{\mathrm{b}}
        $$
        Hãy tính giá trị của biểu thức: $\mathrm{B}=\left(1+\frac{\mathrm{b}}{\mathrm{a}}\right)\left(1+\frac{\mathrm{a}}{\mathrm{c}}\right)\left(1+\frac{\mathrm{c}}{\mathrm{b}}\right)$.
        \item Cho hai đa thức: $f(x)=(x-1)(x+3)$ và $g(x)=x^3-a x^2+b x-3$
        Xác định hệ số $a$; b của đa thức $\mathrm{g}(\mathrm{x})$ biết nghiệm của đa thức $\mathrm{f}(\mathrm{x})$ cũng là nghiệm của đa thức $\mathrm{g}(\mathrm{x})$.
        \item Tìm các số nguyên dương $x, y$, $z$ thỏa mãn: $x+y+z=x y z$.
    \end{enumerate}
	\loigiai{} 
\end{bt}

\begin{bt}
    Cho tam giác $\mathrm{ABC}$ cân tại $\mathrm{A}, \mathrm{BH}$ vuông góc $\mathrm{AC}$ tại $\mathrm{H}$. Trên cạnh $\mathrm{BC}$ lấy điểm $\mathrm{M}$ bất kì ( $M$ khác $B$ và $C)$. Gọi $D, E, F$ là chân đường vuông góc hạ từ $M$ đến $A B, A C, B H$.
	\begin{enumerate}[a.]
        \item Chứng minh: $\triangle \mathrm{DBM}=\triangle \mathrm{FMB}$.
        \item Chứng minh khi $M$ chạy trên cạnh $\mathrm{BC}$ thì tổng $\mathrm{MD}+\mathrm{ME}$ có giá trị không đổi.
        \item Trên tia đối của tia $CA$ lấy điểm $\mathrm{K}$ sao cho $\mathrm{CK}=\mathrm{EH}$.
        Chứng minh $\mathrm{BC}$ đi qua trung điêm của đoạn thẳng $\mathrm{DK}$.
    \end{enumerate}
	\loigiai{}
\end{bt}

\begin{bt}
    Cho tam giác $A B C\left(A B<A C, B=60^{\circ}\right)$. Hai tia phân giác $A D$ $(D \in B C)$ và $C E$ ( $\mathrm{E} \in \mathrm{AB}$ ) của $\triangle \mathrm{ABC}$ cắt nhau ở I. Chứng minh $\Delta \mathrm{IDE}$ cân.
\loigiai{}
\end{bt}

\begin{bt}
    Cho $\operatorname{S}_{\mathrm{n}}=\frac{1^2-1}{1}+\frac{2^2-1}{2^2}+\frac{3^2-1}{3^2}+\ldots+\frac{\mathrm{n}^2-1}{\mathrm{n}^2}$ (với $\mathrm{n} \in \mathrm{N}$ và $\mathrm{n}>1$ )
    
    Chứng minh rằng $\mathrm{S}_{\mathrm{n}}$ không là số nguyên.
\end{bt}

