\onehalfspacing
\section{Đề số 8}

\begin{bt} 
	\hfill
	\begin{enumerate}[a.]
		\item So sánh hai số: $(-5)^{39}$ và $(-2)^{91}$
		\item Chứng minh rằng: Số $\mathrm{A}=11^{\mathrm{n}+2}+12^{2 \mathrm{n}+1}$ chia hết cho 133 , với mọi $\mathrm{n} \in \mathrm{N}$
	\end{enumerate}
	\loigiai{} 
\end{bt}

\begin{bt}
	\hfill
	\begin{enumerate}[a.]
		\item Tìm tất cả các cặp số $(x ; y)$ thỏa mãn: $(2 x-y+7)^{2012}+|x-3|^{2013} \leq 0$
		\item Tìm số tự nhiên $\mathrm{n}$ và chữ số $\mathrm{a}$ biết rằng: $1+2+3+\ldots+n=\overline{a a a}$
	\end{enumerate}
	\loigiai{} 
\end{bt}

\begin{bt}
	Ba lớp 7 ở trường K có tất cả 147 học sinh. Nếu đưa $\frac{1}{3}$ số học sinh của lớp $7 \mathrm{~A}_1, \frac{1}{4}$ số học sinh của lớp $7 \mathrm{~A}_2$ và $\frac{1}{5}$ số học sinh của lớp $7 \mathrm{~A}_3$ đi thi học sinh giỏi cấp huyện thì số học sinh còn lại của ba lớp bằng nhau. Tính tổng số học sinh của mỗi lớp 7 ở trường K.
	\loigiai{} 
\end{bt}

\begin{bt}
	Cho tam giác $\mathrm{ABC}$ có $\hat{A}=3 \hat{B}=6 \hat{C}$.
	\begin{enumerate}[a.]
		\item Tính số đo các góc của tam giác $\mathrm{ABC}$.
		\item Kẻ $\mathrm{AD}$ vuông góc với $\mathrm{BC}$ (D thuộc $\mathrm{BC}$ ). Chứng minh: $\mathrm{AD}<\mathrm{BD}<\mathrm{CD}$.
	\end{enumerate}
	\loigiai{}
\end{bt}

\begin{bt}
	Cho tam giác $A B C$ cân ở $A$. Trên cạnh $A B$ lấy điểm $M$, trên tia đối của tia CA lấy điểm $\mathrm{N}$ sao cho $\mathrm{AM}+\mathrm{AN}=2 \mathrm{AB}$.
	\begin{enumerate}[a.]
		\item Chứng minh rằng: $\mathrm{BM}=\mathrm{CN}$
		\item Chứng minh rằng: $\mathrm{BC}$ đi qua trung điểm của đoạn thẳng $\mathrm{MN}$.
		\item Đường trung trực của $\mathrm{MN}$ và tia phân giác của góc $\mathrm{BAC}$ cắt nhau tại $\mathrm{K}$. Chứng minh rằng: $\mathrm{KC} \perp \mathrm{AC}$.
	\end{enumerate}
	\loigiai{}
\end{bt}