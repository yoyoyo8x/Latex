\onehalfspacing
\section{Đề số 10}

\begin{bt} 
	Tính giá trị của biểu thức
	\begin{enumerate}[a.]
		\item $A=\frac{4^5 \cdot 9^4-2 \cdot 6^9}{2^{10} \cdot 3^8+6^8 \cdot 20}$
		\item $\mathrm{B}=1+3+3^2+3^3+\ldots+3^{2015}-\frac{3^{2016}}{2}$
	\end{enumerate}
	\loigiai{} 
\end{bt}

\begin{bt}
	\hfill
	\begin{enumerate}[a.]
		\item Tìm $x$ biết: $\frac{15}{28}-\left|x-\frac{3}{14}\right|=-\frac{5}{12}$
		\item Tìm $x$, y nguyên biết: $25-y^2=4(x-2016)^2$
	\end{enumerate}
	\loigiai{} 
\end{bt}

\begin{bt}
	\hfill
	\begin{enumerate}[a.]
		\item Cho đa thức: $f(x)=a x^2+b x+c$
		Biết $13 \mathrm{a}+\mathrm{b}+2 \mathrm{c}=0$. Chứng minh $\mathrm{f}(-2) \cdot \mathrm{f}(3) \leq 0$
		\item Cho các số thực $x, y, z \neq 0$ thỏa mãn: $\frac{x y}{x+y}=\frac{y z}{y+z}=\frac{x z}{x+z}$
		Tính giá trị cuả biểu thức: $\mathrm{M}=\frac{\mathrm{x}^2+\mathrm{y}^2+\mathrm{z}^2}{\mathrm{xy}+\mathrm{yz}+\mathrm{xz}}$.
	\end{enumerate}
	\loigiai{} 
\end{bt}

\begin{bt}
	Cho tam giác $\mathrm{ABC}$ vuông ở $\mathrm{A}$, có phân giác $\mathrm{BD}, \mathrm{CE}$ cắt nhau ở $\mathrm{I}$. Gọi $\mathrm{M}, \mathrm{N}$ lân lượt là hình chiếu của $D, E$ trên $B C$
	\begin{enumerate}[a.]
		\item Chứng minh tam giác $\mathrm{ABM}$ cân.
		\item Chứng minh $\mathrm{MN}=\mathrm{AB}+\mathrm{AC}-\mathrm{BC}$
		\item Tính góc MAN.
		\item Gọi $G, K$ lân lượt là giao điểm của $B D$ và $A N ; C E$ và $A M$. Tia $A I$ cắt $G K$ ở $H$. Tính góc AHG.
	\end{enumerate}
	\loigiai{}
\end{bt}
