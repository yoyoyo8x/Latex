\onehalfspacing
\section{Đề số 22}
\graphicspath{{./img/}}
\begin{bt} 
    Thực hiện phép tính:
    $$
    A=1+5+5^2+5^3+5^4+\ldots+5^{2015} B=\frac{4^5 \cdot 9^4-2 \cdot 6^9}{2^{10} \cdot 3^8+6^8 \cdot 20}
    $$
\loigiai{}
\end{bt}

\begin{bt}
    \hfill
	\begin{enumerate}[a.]
        \item Tìm $x$ để biểu thức $\mathrm{P}=1+\frac{9}{3+|x-5|}$ đạt giá trị lớn nhất.
        \item Tìm giá trị của $x$ biết: $\quad|2 x-1|=2$.
        \item Cho 4 số $\mathrm{a}, \mathrm{b}, \mathrm{c}, \mathrm{d}$ trong đó $\mathrm{b}$ là trung bình cộng của a và $\mathrm{c}$ đồng thời $\frac{1}{c}=\frac{1}{2}\left(\frac{1}{b}+\frac{1}{d}\right)$.
        
        Chứng minh bốn số đó lập thành tỉ lệ thức.
    \end{enumerate}
	\loigiai{} 
\end{bt}

\begin{bt}
    Nhà trường thành lập 3 nhóm học sinh khối 7 tham gia chăm sóc di tích lịch sử. Trong đó $\frac{2}{3}$ số học sinh của nhóm I bằng $\frac{8}{11}$ số học sinh của nhóm II và bằng $\frac{4}{5}$ số học sinh của nhóm III. Biết rằng số học sinh của nhóm I ít hơn tổng số học sinh của nhóm II và nhóm III là 18 học sinh. Tính số học sinh của mỗi nhóm.
	\loigiai{}
\end{bt}

\begin{bt}
    Cho $\triangle \mathrm{ABC}$ có $\hat{\mathrm{A}}<90^{\circ}$. Vẽ ra phía ngoài tam giác đó hai đoạn thẳng $\mathrm{AD}$ vuông góc và bằng $A B ; A E$ vuông góc và bằng $A C$.
    \begin{enumerate}[a.]
        \item Chứng minh: $\mathrm{DC}=\mathrm{BE}$ và $\mathrm{DC} \perp \mathrm{BE}$
        \item Gọi $N$ là trung điểm của $DE$. Trên tia đối của tia $NA$ lấy $M$ sao cho $NA=NM$. Chứng minh: $\mathrm{AB}=\mathrm{ME}$ và $\triangle \mathrm{ABC}=\Delta \mathrm{EMA}$.
        \item Chứng minh: $\mathrm{MA} \perp \mathrm{BC}$.
    \end{enumerate}
\loigiai{}
\end{bt}

\begin{bt}
    Một số chính phương có dạng $\overline{a b c d}$. Biết $\overline{a b}-\overline{c d}=1$. Hãy tìm số $\overline{a b c d}$.
\loigiai{}
\end{bt}

