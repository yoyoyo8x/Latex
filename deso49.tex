\section{Đề số 49}

\begin{bt} 
	\hfill
	\begin{enumerate}[a.]
		\item Thực hiện phép tính:
		$$
		A=\frac{2^{12} \cdot 3^5-4^6 \cdot 9^2}{\left(2^2 \cdot 3\right)^6+8^4 \cdot 3^5}+\frac{16^3 \cdot 3^{10}+120 \cdot 6^9}{4^6 \cdot 3^{12}+6^{12}}
		$$
		\item Cho đa thức $P(x)=x^{2012}-2011 x^{2011}-2011 x^{2010}-\ldots . .-2011 x^2-2011 x+1$ \\Tính P( 2012)
		\item Chứng minh rằng : $\frac{1}{7^2}-\frac{1}{7^4}+\ldots+\frac{1}{7^{4 n-2}}-\frac{1}{7^{4 n}}+\ldots+\frac{1}{7^{98}}-\frac{1}{7^{100}}<\frac{1}{50}$
	\end{enumerate}
	\loigiai{} 
\end{bt}

\begin{bt}
	Tìm $x, y, z$ biết :
	\begin{enumerate}[a.]
		\item $2012=|x-2010|+|x-2008|$
		\item $(x-3)^x-(x-3)^{x+2}=0$
		\item $\frac{3 x-2 y}{5}=\frac{2 z-5 x}{3}=\frac{5 y-3 z}{2}$ và $\mathrm{x}+\mathrm{y}+\mathrm{z}=50$
	\end{enumerate}
	\loigiai{} 
\end{bt}

\begin{bt}
	\hfill
	\begin{enumerate}[a.]
		\item Cho dãy tỷ số bằng nhau:
		$$
		\begin{aligned}
			& \frac{2012 a+b+c+d}{a}=\frac{a+2012 b+c+d}{b}=\frac{a+b+2012 c+d}{c}=\frac{a+b+c+2012 d}{d} \\
			& \text { Tính } M=\frac{a+b}{c+d}+\frac{b+c}{d+a}+\frac{c+d}{a+b}+\frac{d+a}{b+c}
		\end{aligned}
		$$
		\item Cho $\mathrm{a}, \mathrm{b}$ là các số nguyên thỏa mãn $(7 \mathrm{a}-21 \mathrm{~b}+5)(\mathrm{a}-3 \mathrm{~b}+1)$ $\vdots$ $ 7$ \\Chứng minh rằng $43 \mathrm{a}+11 \mathrm{~b}+15$ $\vdots$ $ 7$
	\end{enumerate}
	
	\loigiai{} 
\end{bt}

\begin{bt}
	Cho biểu thức : $\mathrm{A}=|x-2010|+|x-2012|+|x-2014|$.
	Tìm $\mathrm{x}$ để biểu thức $\mathrm{A}$ có giá trị nhỏ nhất. Tìm giá trị nhỏ nhất đó .
	\loigiai{}
\end{bt}

\begin{bt}
	Cho tam giác $\mathrm{ABC}$ vuông tại $\mathrm{A}$. M là một điểm thuộc cạnh $\mathrm{BC}$. Qua $\mathrm{M}$ dựng các đoạn thẳng $\mathrm{MD}, \mathrm{ME}$ sao cho $\mathrm{AB}$ là đường trung trực của đoạn thẳng $\mathrm{MD}$ và $\mathrm{AC}$ là đường trung trực của đoạn thẳng ME.
	\begin{enumerate}[a.]
		\item Với điểm $M$ không trùng với điểm $B$ và $C$.
		Chứng minh rằng : $\mathrm{AM}=\mathrm{AD}=\mathrm{AE}$
		\item Với $\mathrm{M}$ bất kỳ . Chứng minh rằng : Ba điểm $\mathrm{A}, \mathrm{D}, \mathrm{E}$ thẳng hàng
		\item Cho tam giác $\mathrm{ABC}$ cố định. Tìm vị trí của điểm $\mathrm{M}$ trên cạnh $\mathrm{BC}$ sao cho $\mathrm{DE}$ có độ dài ngắn nhất .
	\end{enumerate}
	\loigiai{} 
\end{bt}

