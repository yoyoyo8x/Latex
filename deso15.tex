\onehalfspacing
\section{Đề số 15}

\begin{bt} 
    \hfill
	\begin{enumerate}[a.]
		\item Thực hiện phép tính:
        $$
        \mathrm{A}=\frac{9 \cdot 6^9 \cdot 120-4^6 \cdot 9^6}{8^4 \cdot 3^{13}-6^{12}} ; \quad \mathrm{B}=\frac{10}{7 \cdot 12}+\frac{10}{12 \cdot 17}+\frac{10}{17 \cdot 22}+\ldots+\frac{10}{2012 \cdot 2017}+\frac{10}{2017 \cdot 2022}
        $$
        \item Cho a, b, c là ba số thực khác 0 , thoả mãn : $\frac{a+b-c}{c}=\frac{b+c-a}{a}=\frac{a+c-b}{b}$.
        Hãy tính giá trị của biểu thức $B=\left(1+\frac{b}{a}\right) \cdot\left(1+\frac{a}{c}\right) \cdot\left(1+\frac{c}{b}\right)$.
        \item Tính giá trị của đa thức $f(x)=x^5-2018 x^4+2016 x^3+2018 x^2-2016 x-2017$ tại $x=2017$
	\end{enumerate}
	\loigiai{} 
\end{bt}

\begin{bt}
	\hfill
	\begin{enumerate}[a.]
		\item Tìm các cặp số tự nhiên $(\mathrm{x} ; \mathrm{y})$ sao cho: $49-\mathrm{y}^2=12(\mathrm{x}-2001)^2$
        \item Cho $\left|2019 x_1-2018 y_1\right|+\left|2019 x_2-2018 y_2\right|+\ldots+\left|2019 x_{2018}-2018 y_{2018}\right| \leq 0$. Chứng minh
        $$
        \frac{x_1+x_2+x_3+\ldots+x_{2018}}{y_1+y_2+y_3+\ldots+y_{2018}}=\frac{2018}{2019} \text {. }
        $$
        \item Một cửa hàng có ba cuộn vải, tổng chiều dài ba cuộn vải đó là $186 \mathrm{~m}$, giá tiền mỗi mét vải của ba cuộn là như nhau. Sau khi bán được một ngày cửa hàng còn lại $\frac{2}{3}$ cuộn thứ nhất, $\frac{1}{3}$ cuộn thứ hai, $\frac{3}{5}$ cuộn thứ ba. Số tiền bán được của ba cuộn thứ nhất, thứ hai, thứ ba lần lượt tỉ lệ với $2 ; 3 ; 2$. Tính xem trong ngày đó cửa hàng đã bán được bao nhiêu mét vải mỗi cuộn.
	\end{enumerate}
	\loigiai{} 
\end{bt}

\begin{bt}
	Cho tam giác $A B C, M$ là trung điểm của $B C$. Trên tia đối của của tia MA lấy điểm $\mathrm{E}$ sao cho $\mathrm{ME}=\mathrm{MA}$. Chứng minh rằng:
	\begin{enumerate}[a.]
		\item $\mathrm{AC}=\mathrm{EB}$ và $\mathrm{AC} / / \mathrm{BE}$
        \item Gọi I là một điểm trên $\mathrm{AC} ; \mathrm{K}$ là một điểm trên $\mathrm{EB}$ sao cho $\mathrm{AI}=\mathrm{EK}$. Chứng minh ba điểm I, M $\mathrm{K}$ thẳng hàng
        \item Từ E kẻ $E H \perp B C(H \in B C)$. Biết $H B E=50^{\circ} ; M E B=25^{\circ}$. Tính $H E M$ và $B M E$
    \end{enumerate}
	\loigiai{}
\end{bt}

\begin{bt}
    Tìm các số tự nhiên $x, y, z \neq 0$ thoả mãn điều kiện: $x+y+z=x y z$
\loigiai{}
\end{bt}

