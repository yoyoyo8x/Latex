\section{Đề số 46}

\begin{bt} 
	Tìm x biết:
	\begin{enumerate}[a.]
		\item $3^{x-1}+5.3^{x-1}=162$
		\item $3 x+x^2=0$
		\item $(x-1)(x-3)<0$
	\end{enumerate}
	\loigiai{} 
\end{bt}

\begin{bt}
	\hfill
	\begin{enumerate}[a.]
		\item Tìm ba số $x, y, z$ thỏa mãn: $\frac{x}{3}=\frac{y}{4}=\frac{z}{5}$ và $2 x^2+2 y^2-3 z^2=-100$
		\item Cho $\frac{a}{2 b}=\frac{b}{2 c}=\frac{c}{2 d}=\frac{d}{2 a}(a, b, c, d>0)$
		\\Tính $\mathrm{A}=\frac{2011 \mathrm{a}-2010 \mathrm{~b}}{\mathrm{c}+\mathrm{d}}+\frac{2011 \mathrm{~b}-2010 \mathrm{c}}{\mathrm{a}+\mathrm{d}}+\frac{2011 \mathrm{c}-2010 \mathrm{~d}}{\mathrm{a}+\mathrm{b}}+\frac{2011 \mathrm{~d}-2010 \mathrm{a}}{\mathrm{b}+\mathrm{c}}$
	\end{enumerate}
	\loigiai{} 
\end{bt}

\begin{bt}
	\hfill
	\begin{enumerate}[a.]
		\item Tìm cặp số nguyên $(x, y)$ thoả mãn $x+y+x y=2$.
		\item Tìm giá trị lớn nhất của biểu thức $Q=\frac{27-2 x}{12-x}$ (với x nguyên)
	\end{enumerate}
	
	\loigiai{} 
\end{bt}

\begin{bt}
	\hfill
	\begin{enumerate}[a.]
		\item  Cho đa thức $f(x)=a x^2+b x+c$. Chứng minh rằng nếu $f(x)$ nhận 1 và $-1$ là nghiệm thì a và c là 2 số đối nhau.
		\item Tìm giá trị nhỏ nhất của biểu thức $\mathrm{P}=(|x-3|+2)^2+|y+3|+2007$
	\end{enumerate}
	\loigiai{} 
\end{bt}

\begin{bt}
	Cho $\triangle \mathrm{ABC}$ vuông tại $\mathrm{A}$. $\mathrm{M}$ là trung điểm $\mathrm{BC}$, trên tia đối của tia MA lấy điểm $\mathrm{D}$ sao cho $\mathrm{AM}=\mathrm{MD}$. Gọi $\mathrm{I}$ và $\mathrm{K}$ lần lượt là chân đường vuông góc hạ từ $\mathrm{B}$ và $\mathrm{C}$ xuống $\mathrm{AD}, \mathrm{N}$ là chân đường vuông góc hạ từ $\mathrm{M}$ xuống $\mathrm{AC}$.
	\begin{enumerate}[a.]
		\item Chứng minh rằng $\mathrm{BK}=\mathrm{CI}$ và $\mathrm{BK} / / \mathrm{CI}$.
		\item Chứng minh $\mathrm{KN}<\mathrm{MC}$.
		\item $\triangle \mathrm{ABC}$ thỏa mãn thêm điều kiện gì để $\mathrm{AI}=\mathrm{IM}=\mathrm{MK}=\mathrm{KD}$.
		\item Gọi $\mathrm{H}$ là chân đường vuông góc hạ từ $\mathrm{D}$ xuống $\mathrm{BC}$. Chứng minh rằng các đường thẳng $\mathrm{BI}, \mathrm{DH}, \mathrm{MN}$ đồng quy.
	\end{enumerate}
	\loigiai{}
\end{bt}

