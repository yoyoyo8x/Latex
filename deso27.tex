\onehalfspacing
\section{Đề số 27}

\begin{bt} 
   \hfill
   \begin{enumerate}[a.]
    \item Thực hiện phép tính: $A=\frac{2^{12} \cdot 3^5-4^6 \cdot 9^2}{\left(2^2 \cdot 3\right)^6+8^4 \cdot 3^5}$
    \item Cho hàm số $y=f(x)=a x^2+b x+c$.
    
    Cho biết $f(0)=2014 ; f(1)=2015 ; f(-1)=2017$. Tính $f(-2)$.
   \end{enumerate}
\loigiai{}
\end{bt}

\begin{bt}
   Tìm $x$, $y$ biết:
	\begin{enumerate}[a.]
        \item $\left|x+\frac{1}{5}\right|-4=-2$
        \item $2^{x-1}+5.2^{x-2}=\frac{7}{32}$
        \item $|x+5|+(3 y-4)^{2016}=0$
        \item $\frac{x}{2}=\frac{y}{5}$ và $x y=40$
    \end{enumerate}
	\loigiai{} 
\end{bt}

\begin{bt}
    \hfill
    \begin{enumerate}[a.]
        \item Tìm tất cả các cặp số nguyên $x$, $y$ sao cho: $2 x y+x-2 y=4$
        \item Số $\mathrm{M}$ được chia thành ba số tỉ lệ với 0,$5 ; 1 \frac{2}{3} ; 2 \frac{1}{4}$. Tìm số $\mathrm{M}$ biết rằng tổng bình phương của ba số đó bằng 4660 .
    \end{enumerate}
	\loigiai{}
\end{bt}

\begin{bt}
    Cho tam giác $\mathrm{ABC}$ cân tại $\mathrm{A}$. Trên cạnh $\mathrm{BC}$ lấy điểm $\mathrm{D}$, trên tia đối của tia $C B$ lấy điểm $E$ sao cho $C E=B D$. Đường thẳng vuông góc với $B C$ kẻ từ $\mathrm{D}$ cắt $A B$ tại $M$. Đường vuông góc với $\mathrm{BE}$ tại $\mathrm{E}$ cắt $\mathrm{AC}$ tại $\mathrm{N}$.
    \begin{enumerate}[a.]
        \item Chứng minh: $\triangle M B D=\triangle N C E$.
        \item Cạnh $\mathrm{BC}$ cắt $\mathrm{MN}$ tại $\mathrm{I}$. Chứng minh $\mathrm{I}$ trung điểm của $\mathrm{MN}$.
        \item Chứng minh đường thẳng vuông góc với $\mathrm{MN}$ tại $\mathrm{I}$ luôn đi qua một điểm cố định khi D thay đổi trên đoạn BC.
    \end{enumerate}
	\loigiai{}
\end{bt}

\begin{bt}
   \hfill
   \begin{enumerate}[a.]
    \item Tìm số tự nhiên có ba chữ số. Biết rằng số đó chia hết cho 7 và tổng các chữ số đó bằng 14.
    \item Cho tam giác $\mathrm{ABC}$ có $B A C=B C A=80^{\circ}$. Ở miền trong của tam giác vẽ hai tia $\mathrm{Ax}$ và $C y$ cắt $\mathrm{BC}$ và $\mathrm{BA}$ lân lượt tại $\mathrm{D}$ và $\mathrm{E}$. Cho biết $C A D=60^{\circ} ; E C A=50^{\circ}$.
    Tính số đo góc $A D E$.
   \end{enumerate}
\loigiai{}
\end{bt}


