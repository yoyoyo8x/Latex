\onehalfspacing
\section{Đề số 1}

\begin{bt}
    \hfill
    \begin{enumerate}[a.]
        \item Thực hiện phép tính: $\frac{0,375-0,3+\frac{3}{11}+\frac{3}{12}}{-0,265+0,5-\frac{5}{11}-\frac{5}{12}}+\frac{1,5+1-0,75}{2,5+\frac{5}{3}-1,25}$
        \item So sánh: $\sqrt{50}+\sqrt{26}+1 \quad$ và $\sqrt{168}$.
    \end{enumerate}
\loigiai{} 
\end{bt}

\begin{bt}
    \hfill
    \begin{enumerate}[a.]
        \item Tìm $x$ biết: $|x-2|+|3-2 x|=2 x+1$
        \item Tìm $x ; y \in Z$ biết: $x y+2 x-y=5$
        \item Tìm x; y; z biết: $2 x=3 y$ ; $4 y=5 z$ và $4 x-3 y+5 z=7$
    \end{enumerate}
\loigiai{} 
\end{bt}

\begin{bt}
    \hfill
    \begin{enumerate}[a.]
        \item Tìm đa thức bậc hai biết $f(x)-f(x-1)=x$.
        
        Từ đó áp dụng tính tổng $\mathrm{S}=1+2+3+\ldots+\mathrm{n}$.
        \item Cho $\frac{2 b z-3 c y}{a}=\frac{3 c x-a z}{2 b}=\frac{a y-2 b x}{3 c}$
        
        Chứng minh: $\frac{x}{a}=\frac{y}{2 b}=\frac{z}{3 c}$.
    \end{enumerate}
\loigiai{} 
\end{bt}

\begin{bt}
    \hfill
    Cho tam giác $\mathrm{ABC}\left(B A C<90^{\circ}\right)$, đường cao $\mathrm{AH}$. Gọi $\mathrm{E} ; \mathrm{F}$ lần lượt là điểm đối xứng của $\mathrm{H}$ qua $\mathrm{AB} ; \mathrm{AC}$, đường thẳng $\mathrm{EF}$ cắt $\mathrm{AB} ; \mathrm{AC}$ lần lượt tại $\mathrm{M}$ và $\mathrm{N}$. Chứng minh rằng:
    \begin{enumerate}[a.]
    \item $\mathrm{AE}=\mathrm{AF}$;
    \item HA là phân giác của $M H N$;
    \item $\mathrm{CM} / / \mathrm{EH} ; \mathrm{BN} / / \mathrm{FH}$.
    \end{enumerate}
\loigiai{}
\end{bt}