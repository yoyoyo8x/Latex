\section{Đề số 48}

I. Phần trắc nghiệm khách quan:
\begin{bt} 
	Giá trị của $x$ trong biểu thức $(\sqrt{x}-1)^2=0,25$ là:
	\begin{enumerate}[A.]
		\item  $\frac{9}{4} ; \frac{1}{4}$
		\item $-\frac{1}{4} ;-\frac{9}{4}$
		\item $\frac{9}{4} ;-\frac{1}{4}$
		\item $-\frac{9}{4} ; \frac{1}{4}$
	\end{enumerate}
	\loigiai{} 
\end{bt}

\begin{bt}
	Cho góc $x \mathrm{Oy}=50^{\circ}$, điểm $\mathrm{A}$ nằm trên $\mathrm{Oy}$. Qua $\mathrm{A}$ vẽ tia $\mathrm{Am}$. Để Am song song với Ox thì số đo của góc $\mathrm{OAm}$ là:
	\begin{enumerate}[a.]
		\item $50^{\circ}$
		\item $130^{\circ}$
		\item $50^{\circ}$ và $130^{\circ}$
		\item $80^{\circ}$
	\end{enumerate}
	\loigiai{} 
\end{bt}

\begin{bt}
	Cho hàm số $\mathrm{y}=\mathrm{f}(\mathrm{x})$ xác định với mọi $x>1$. Biết $\mathrm{f}(\mathrm{n})=(\mathrm{n}-1) \cdot \mathrm{f}(\mathrm{n}-1)$ và $\mathrm{f}(1)=1$. Giá trị của $\mathrm{f}(4)$ là:
	\begin{enumerate}[a.]
		\item 3
		\item 5
		\item 6
		\item 1
	\end{enumerate}
	\loigiai{} 
\end{bt}

\begin{bt}
	Cho tam giác $A B C$ vuông tại $B, A B=6, \hat{A}=30^{\circ}$. Phân giác góc $C$ cắt $A B$ tại $D$. Khi đó độ dài đoạn thẳng $\mathrm{BD}$ và $\mathrm{AD}$ lần lượt là:
	\begin{enumerate}[a.]
		\item $2 ; 4$
		\item $3 ; 3$
		\item $4 ; 2$
		\item $1 ; 5$
	\end{enumerate}
	\loigiai{} 
\end{bt}

\begin{bt}
	Cho $a^{2 \mathrm{~m}}=-4$. Kết quả của $2 a^{6 m}-5$ là:
	\begin{enumerate}[a.]
		\item $-123$
		\item $-133$
		\item 123
		\item $-128$
	\end{enumerate}
	\loigiai{} 
\end{bt}

\begin{bt}
	Cho tam giác $\mathrm{DEF}$ có $\angle \mathrm{E}=\angle \mathrm{F}$. Tia phân giác của góc $\mathrm{D}$ cắt EF tại $\mathrm{I}$. Ta có:
	\begin{enumerate}[a.]
		\item $\triangle \mathrm{DIE}=\Delta \mathrm{DIF}$
		\item $\mathrm{DE}=\mathrm{DF}, \angle \mathrm{IDE}=\angle \mathrm{IDF}$
		\item $\mathrm{IE}=\mathrm{IF} ; \mathrm{DI}=\mathrm{EF}$
		\item Cả $\mathrm{A}, \mathrm{B}, \mathrm{C}$ đều đúng
	\end{enumerate}
	\loigiai{} 
\end{bt}

\begin{bt}
	Biết $\mathrm{a}+\mathrm{b}=9$. Kết quả của phép tính $0, a(b)+0, b(a)$ là:
	\begin{enumerate}[a.]
		\item 2
		\item 1
		\item 0,5
		\item 1,5
	\end{enumerate}
	\loigiai{} 
\end{bt}

\begin{bt}
	Cho $(a-b)^2+6 a \cdot b=36$. Giá trị lớn nhất của $x=a \cdot b$ là:
	\begin{enumerate}[a.]
		\item 6
		\item $-6$
		\item 7
		\item 5
	\end{enumerate}
	\loigiai{} 
\end{bt}

\begin{bt}
	Cho tam giác $A B C$, hai đường trung tuyến $\mathrm{BM}, \mathrm{CN}$. Biết $\mathrm{AC}>\mathrm{AB}$. Khi đó độ dài hai đoạn thẳng $\mathrm{BM}$ và $C N$ là:
	\begin{enumerate}[a.]
		\item $\mathrm{BM} \leq \mathrm{CN}$
		\item $\mathrm{BM}>\mathrm{CN}$
		\item $\mathrm{BM}<\mathrm{CN}$
		\item $\mathrm{BM}=\mathrm{CN}$
	\end{enumerate}
	\loigiai{} 
\end{bt}

\begin{bt}
	Điểm thuộc đồ thị hàm số $\mathrm{y}=-2 \mathrm{x}$ là:
	\begin{enumerate}[a.]
		\item M $(-1 ;-2)$
		\item $\mathrm{N}(1 ; 2) \quad$
		\item $\mathrm{P}(0 ;-2)$
		\item $Q(-1 ; 2)$
	\end{enumerate}
	\loigiai{} 
\end{bt}

\begin{bt}
	Biết rằng lãi suất hàng năm của tiền gửi tiết kiệm theo mức $5 \%$ năm là một hàm số theo số tiên gửi: $\mathrm{i}=0,005 \mathrm{p}$. Nếu tiền gửi là 175000 thì tiền lãi sẽ là:
	\begin{enumerate}[a.]
		\item 8850 đ
		\item 8750 đ
		\item 7850 đ
		\item 7750 đ
	\end{enumerate}
	\loigiai{} 
\end{bt}

\begin{bt}
	Cho tam giác $A B C$ cân tại $A, \hat{A}=20^{\circ}$. Trên cạnh $A B$ lấy điểm $D$ sao cho $A D=B C$. Số đo của góc BDC là:
	\begin{enumerate}[a.]
		\item $50^{\circ}$
		\item $70^{\circ}$
		\item $30^{\circ}$
		\item $80^{\circ}$
	\end{enumerate}
	\loigiai{} 
\end{bt}

II. Phần tự luận
\begin{bt}
	\hfill
	\begin{enumerate}[a.]
		\item Chứng tỏ rằng: $M=75 \cdot\left(4^{2017}+4^{2016}+\ldots+4^2+4+1\right)+25$ chia hết cho $10^2$
		\item Cho tích $a \cdot b$ là số chính phương và $(\mathrm{a}, \mathrm{b})=1$. Chứng minh rằng $\mathrm{a}$ và $\mathrm{b}$ đều là số chính phương.
	\end{enumerate}
	
	\loigiai{} 
\end{bt}


\begin{bt}
	\hfill
	\begin{enumerate}[1.]
		\item Cho đa thức $\mathrm{A}=2 \mathrm{x} .(\mathrm{x}-3)-\mathrm{x}(\mathrm{x}-7)-5(\mathrm{x}-403)$
		\\Tính giá trị của $\mathrm{A}$ khi $\mathrm{x}=4$. Tìm $\mathrm{x}$ để $\mathrm{A}=2015$
		\item Học sinh khối 7 của một trường gồm 3 lớp tham gia trồng cây. Lớp 7A trồng toàn bộ $32,5 \%$ số cây. Biết số cây lớp $7 \mathrm{~B}$ và $7 \mathrm{C}$ trồng được theo tỉ lệ 1,5 và 1,2 . Hỏi số cây cả 3 lớp trồng được là bao nhiêu, biết số cây của lớp $7 \mathrm{~A}$ trồng được ít hơn số cây của lớp 7B trồng được là 120 cây.
	\end{enumerate}
	\loigiai{} 
\end{bt}


\begin{bt}
	\hfill
	\begin{enumerate}[1.]
		\item Cho đoạn thẳng $\mathrm{AB}$. Trên cùng một nửa mặt phẳng có bờ là đường thẳng $\mathrm{AB}$ vẽ hai tia $\mathrm{Ax}$ và $B y$ lần lượt vuông góc với $\mathrm{AB}$ tại $\mathrm{A}$ và $\mathrm{B}$. Gọi $\mathrm{O}$ là trung điểm của đoạn thẳng $\mathrm{AB}$. Trên tia $A x$ lấy điểm $C$ và trên tia By lấy điểm $\mathrm{D}$ sao cho góc $C O D$ bằng $90^{\circ}$.
		\begin{enumerate}[a.]
			\item Chứng minh rằng: $\mathrm{AC}+\mathrm{BD}=\mathrm{CD}$.
			\item Chứng minh rằng: $A C \cdot B D=\frac{A B^2}{4}$
		\end{enumerate}
		\item Cho tam giác nhọn $\mathrm{ABC}$, trực tâm $\mathrm{H}$. Chứng minh rằng:
		$$
		\mathrm{HA}+\mathrm{HB}+\mathrm{HC}<\frac{2}{3}(A B+A C+B C)
		$$
	\end{enumerate}
	\loigiai{}
\end{bt}

\begin{bt}
	Tìm giá trị nhỏ nhất của $\mathrm{A}$, biết:
	$$
	A=|7 x-5 y|+|2 z-3 x|+|x y+y z+z x-2000|
	$$
	\loigiai{} 
\end{bt}


