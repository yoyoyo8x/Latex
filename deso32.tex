\section{Đề số 32}

\begin{bt} 
	\hfill
	\begin{enumerate}[a.]
		\item Tính giá trị của biểu thức $\mathrm{A}=\left(\frac{-4}{7}+\frac{2}{5}\right): \frac{2}{3}+\left(\frac{-3}{7}+\frac{3}{5}\right): \frac{2}{3}$
		\item Tính giá trị của biểu thức $\mathrm{B}=2 \mathrm{x}^2-3 \mathrm{x}+1$ với $|x|=\frac{1}{2}$.
		\item Tìm 3 số $x, y, z$ biết rằng: $\frac{x}{3}=\frac{y}{7} ; \frac{y}{2}=\frac{z}{5}$ và $x+y+z=-110$.
	\end{enumerate}
	\loigiai{} 
\end{bt}

\begin{bt}
	\hfill
	\begin{enumerate}[a.]
		\item Tìm tập hợp các số nguyên $x$, biết rằng:
		$$
		4 \frac{5}{9}: 2 \frac{5}{18}-7<x<\left(3 \frac{1}{5}: 3,2+4,5.1 \frac{31}{45}\right):\left(-21 \frac{1}{2}\right)
		$$
		\item Cho $\frac{a}{c}=\frac{c}{b}$. Chứng minh rằng: $\frac{a^2+c^2}{b^2+c^2}=\frac{a}{b}$
		\item Tính giá trị của biểu thức: $C=2 x^5-5 y^3+2015$ tại $x, y$ thỏa mãn:
		$$	|x-1|+(y+2)^{20}=0 $$
	\end{enumerate}
	\loigiai{} 
\end{bt}

\begin{bt}
	\hfill 
	\begin{enumerate}[a.]
		\item Tìm số tự nhiên có ba chữ số, biết rằng số đó là bội của 18 và các chữ số của nó tỉ lệ theo 1: 2: 3.
		\item Tìm tất cả các số tự nhiên $a, b$ sao cho : $2^{\mathrm{a}}+37=|b-45|+\mathrm{b}-45$.
	\end{enumerate}
	\loigiai{} 
\end{bt}

\begin{bt}
	Cho tam giác $\mathrm{ABC}$ có ba góc nhọn $(\mathrm{AB}<\mathrm{AC})$. Vẽ về phía ngoài tam giác $A B C$ các tam giác đều $A B D$ và $A C E$. Gọi $I$ là giao của $C D$ và $B E, K$ là giao của $A B$ và $D C$.
	\begin{enumerate}[a.]
		\item Chứng minh rằng: $\triangle \mathrm{ADC}=\triangle \mathrm{ABE}$.
		\item Chứng minh rằng: góc $\mathrm{DIB}=60^{\circ}$.
		\item Gọi $\mathrm{M}$ và $\mathrm{N}$ lân lượt là trung điểm của $\mathrm{CD}$ và $\mathrm{BE}$. Chứng minh rằng $\triangle \mathrm{AMN}$ đều.
		\item Chứng minh rằng IA là phân giác của góc DIE.
	\end{enumerate}
	\loigiai{}
\end{bt}

\begin{bt}
	Cho 20 số nguyên khác $0: a_1, a_2, a_3, \ldots, a_{20} $ có các tính chất sau: 
	\begin{itemize}[*]
		\item $a_1$ là số dương.
		\item Tổng của ba số viết liền nhau bất kì là một số dương.
		\item Tổng của 20 số đó là số âm.
	\end{itemize}
	Chứng minh rằng : $a_{1} . a_{14}+ a_{14} . a_{12}<a_1 \cdot a_{12}$.
	\loigiai{}
\end{bt}
