\section{Đề số 36}

\begin{bt} 
	\hfill
	\begin{enumerate}[a.]
		\item Cho a, b, c là ba số thực dương thỏa mãn điều kiện: $\frac{a+b-c}{c}=\frac{b+c-a}{a}=\frac{c+a-b}{b}$. \\Hãy tính giá trị của biểu thức: $B=\left(1+\frac{b}{a}\right)\left(1+\frac{a}{c}\right)\left(1+\frac{c}{b}\right)$.
		\item Cho tỉ lệ thức $\frac{a}{b}=\frac{c}{d}$ với $a \neq 0, b \neq 0, c \neq 0, d \neq 0, a \neq \pm b, c \neq \pm d$.
		\\Chứng minh: $\left(\frac{a-b}{c-d}\right)^{2013}=\frac{a^{2013}+b^{2013}}{c^{2013}+d^{2013}}$
	\end{enumerate}
	
	\loigiai{} 
\end{bt}

\begin{bt}
	\hfill
	\begin{enumerate}[a.]
		\item Cho $\frac{x}{y+z+t}=\frac{y}{z+t+x}=\frac{z}{t+x+y}=\frac{t}{x+y+z}$
		\\Chứng minh rằng: Biểu thức sau có giá trị nguyên
		$$
		A=\frac{x+y}{z+t}+\frac{y+z}{t+x}+\frac{z+t}{x+y}+\frac{t+x}{y+z}
		$$
		\item Tìm $x$ biết: $x^2-5 x+6=0$
		\item Số $\mathrm{A}$ được chia thành ba phần số tỉ lệ theo $\frac{2}{5}: \frac{3}{4}: \frac{1}{6}$. Biết rằng tổng các bình phương của ba số đó bằng 24309 . Tìm số $\mathrm{A}$.
	\end{enumerate}
	\loigiai{} 
\end{bt}

\begin{bt}
	Tìm giá trị nhỏ nhất của biểu thức: $A=|x-2013|+|x-3014|+|x-2015|$
	\loigiai{} 
\end{bt}

\begin{bt}
	Tìm hai số dương biết tổng, hiệu, tích của chúng tỉ lệ nghịch với ba số 20;120;16.
	\loigiai{}
\end{bt}

\begin{bt}
	Cho tam giác $\mathrm{ABC}$ vuông ở $\mathrm{A}$, có góc $C=30^{\circ}$, đường cao $\mathrm{AH}$. Trên đoạn $\mathrm{HC}$ lấy điểm $\mathrm{D}$ sao cho $H D=H B$. Từ $\mathrm{C}$ kẻ $\mathrm{CE}$ vuông góc với $\mathrm{AD}$. Chứng minh:
	\begin{enumerate}[a.]
		\item Tam giác $\mathrm{ABD}$ là tam giác đều.
		\item $A H=C E$.
		\item HE song song với $A C$.
	\end{enumerate}
	\loigiai{}
\end{bt}


