\section{Đề số 28}

\begin{bt} 
   \hfill
   \begin{enumerate}[a.]
    \item Thực hiện phép tính: $\quad B=\frac{1}{-77^2} \cdot 7^4(-11)^2 \cdot 77^5 \cdot\left(\frac{1}{7^2}\right)^2:\left(7^3 \cdot 11^6\right)$
    \item Cho các số $a, b, c$ khác 0 thỏa mãn: $\frac{a-b+c}{2 b}=\frac{c-a+b}{2 a}=\frac{a-c+b}{2 c}$
    
    Tính giá trị biểu thức: $\mathrm{P}=\left(1+\frac{\mathrm{c}}{\mathrm{b}}\right) \cdot\left(1+\frac{\mathrm{b}}{\mathrm{a}}\right) \cdot\left(1+\frac{\mathrm{a}}{\mathrm{c}}\right)$
   \end{enumerate}
\loigiai{}
\end{bt}

\begin{bt}
    \hfill
	\begin{enumerate}[a.]
        \item Tìm $x$ biết: $\frac{2}{|x-2|+2}=\frac{3}{|6-3 x|+1}$
        \item Tìm hình chữ nhật có kích thước các cạnh là số nguyên sao cho số đo diện tích bằng số đo chu vi.
        \item Tìm các số nguyên dương x; y; z thỏa mãn:
        $$
        (x-y)^3+(y-z)^2+2015 \cdot|x-z|=2017
        $$
    \end{enumerate}
	\loigiai{} 
\end{bt}

\begin{bt}
    Cho hàm số: $\mathrm{y}=\mathrm{f}(\mathrm{x})=\mathrm{x}+\frac{3}{2}|\mathrm{x}|(1)$
    \begin{enumerate}[a.]
        \item Vẽ đồ thị hàm số (1).
        \item Gọi $E$ và $F$ là hai điểm thuộc đồ thị hàm số (1) có hoành độ lần lượt là (-4) và $\frac{4}{5}$, xác định tọa độ hai điểm $\mathrm{E}, \mathrm{F}$. Tìm trên trục tung điểm $\mathrm{M}$ để $EM+MF$ nhỏ nhất.
    \end{enumerate}
	\loigiai{}
\end{bt}

\begin{bt}
    \hfill
    \begin{enumerate}[1.]
        \item Cho tam giác $A B C$ nhọn; vẽ về phía ngoài tam giác $A B C$ các tam giác vuông cân tại $\mathrm{A}$ là tam giác $\mathrm{ABD}$ và tam giác $\mathrm{ACE}$.
        \begin{enumerate}[a.]
            \item Chứng minh $\mathrm{DC}=\mathrm{BE}$ và $\mathrm{DC} \perp \mathrm{BE}$.
            \item Gọi $\mathrm{H}$ là chân đường vuông góc kẻ từ $\mathrm{A}$ đến $\mathrm{ED}$ và $\mathrm{M}$ là trung điểm của đoạn thẳng $\mathrm{BC}$. Chứng minh $\mathrm{A}, \mathrm{M}, \mathrm{H}$ thẳng hàng .
        \end{enumerate}
        \item Cho tam giác $\mathrm{ABC}$ vuông tại $\mathrm{A}$ có $\mathrm{AB}=3 \mathrm{~cm} ; \mathrm{AC}=4 \mathrm{~cm}$. Điểm $I$ nằm trong tam giác và cách đều ba cạnh của tam giác $\mathrm{ABC}$. Gọi $\mathrm{M}$ là chân đường vuông góc kẻ từ điểm $I$ đến $BC$. Tính $MB$.
    \end{enumerate}
	\loigiai{}
\end{bt}

\begin{bt}
    Chứng minh rằng với mọi số tự nhiên $\mathrm{n} \geq 2$ thì tổng:
    $$
    \mathrm{S}=\frac{3}{4}+\frac{8}{9}+\frac{15}{16}+\ldots+\frac{\mathrm{n}^2-1}{\mathrm{n}^2} \text { không thể là một số nguyên. }
    $$
\loigiai{}
\end{bt}


