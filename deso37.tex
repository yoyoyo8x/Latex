\onehalfspacing
\section{Đề số 37}

\begin{bt} 
	\hfill
	\begin{enumerate}[a.]
		\item Tính giá trị biểu thức $\mathrm{P}=\left|a-\frac{1}{2014}\right|+\left|a-\frac{1}{2016}\right|$, với $a=\frac{1}{2015}$.
		\item Tìm số nguyên $\mathrm{x}$ để tích hai phân số $\frac{6}{x+1}$ và $\frac{x-1}{3}$ là một số nguyên.
	\end{enumerate}
	\loigiai{} 
\end{bt}

\begin{bt}
	\hfill
	\begin{enumerate}[a.]
		\item Cho $\mathrm{a}>2, \mathrm{~b}>2$. Chứng minh $a b>a+b$
		\item Cho ba hình chữ nhật, biết diện tích của hình thứ nhất và diện tích của hình thứ hai tỉ lệ với 4 và 5 , diện tích hình thứ hai và diện tích hình thứ ba tỉ lệ với 7 và 8 , hình thứ nhất và hình thứ hai có cùng chiều dài và tổng các chiều rộng của chúng là $27 \mathrm{~cm}$, hình thứ hai và hình thứ ba có cùng chiều rộng, chiều dài của hình thứ ba là $24 \mathrm{~cm}$. Tính diện tích của mỗi hình chữ nhật đó.
	\end{enumerate}
	\loigiai{} 
\end{bt}

\begin{bt}
	Cho tam giác $A B C, M$ là trung điểm của $B C$. Trên tia đối của tia MA lấy điểm $E$ sao cho $\mathrm{ME}=\mathrm{MA}$. Chứng minh rằng:
	\begin{enumerate}[a.]
		\item AC = EB và AC // BE
		\item Gọi I là một điểm trên $\mathrm{AC}, \mathrm{K}$ là một điểm trên $\mathrm{EB}$ sao cho: $\mathrm{AI}=\mathrm{EK}$. Chứng minh: $\mathrm{I}, \mathrm{M}, \mathrm{K}$ thẳng hàng.
		\item Từ $\mathrm{E}$ kẻ $\mathrm{EH} \perp \mathrm{BC}(\mathrm{H} \in \mathrm{BC})$. Biết góc $\mathrm{HBE}$ bằng $50^{\circ}$; góc $\mathrm{MEB}$ bằng $25^{\circ}$, tính các góc $H E M$ và $B M E$ ?
	\end{enumerate}
	\loigiai{}
\end{bt}

\begin{bt}
	Cho các số $0<a_1<a_2<a_3<\ldots . .<a_{15}$. Chứng minh rằng $\frac{a_1+a_2+a_3+\ldots+a_{15}}{a_5+a_{10}+a_{15}}<5$
	\loigiai{}
\end{bt}

\begin{bt}
	Cho $\triangle \mathrm{ABC}$ nhọn với $B A C=60^{\circ}$. Chứng minh rằng:
	$$
	\mathrm{BC}^2=\mathrm{AB}^2+\mathrm{AC}^2-\mathrm{AB} \cdot \mathrm{AC}
	$$
	\loigiai{} 
\end{bt}