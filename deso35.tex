\onehalfspacing
\section{Đề số 35}

\begin{bt} 
	\hfill
	\begin{enumerate}[1.]
		\item Tính giá trị của biểu thức
		$$
		\begin{aligned}
			& \mathrm{A}=(-1)^3 \cdot\left(-\frac{7}{8}\right)^3 \cdot\left(-\frac{2}{7}\right)^2 \cdot(-7) \cdot\left(-\frac{1}{14}\right) \\
			& \mathrm{B}=2016:\left(\frac{0,4-\frac{2}{9}+\frac{2}{11}}{1,4-\frac{7}{9}+\frac{7}{11}} \cdot \frac{-1 \frac{1}{6}+0,875-0,7}{\frac{1}{3}-0,25+\frac{1}{5}}\right)
		\end{aligned}
		$$
		\item Cho đa thức $Q(x)=a x^3+b x^2+c x+d$ với $a, b, c, d \in Z$. Biết $Q(x)$ chia hết cho 3 với mọi $x \in Z$. Chứng tỏ các hệ số $a, b, c, d$ đều chia hết cho 3.
	\end{enumerate}
	\loigiai{} 
\end{bt}

\begin{bt}
	\hfill
	\begin{enumerate}[1.]
		\item Biết $\frac{b z-c y}{a}=\frac{c x-a z}{b}=\frac{a y-b x}{c}($ với $a, b, c \neq 0)$.
		Chứng minh rằng: $\frac{\mathrm{x}}{\mathrm{a}}=\frac{\mathrm{y}}{\mathrm{b}}=\frac{\mathrm{z}}{\mathrm{c}}$.
		\item Số $\mathrm{M}$ được chia thành ba phân tỉ lệ nghịch với $3 ; 5 ; 6$. Biết rằng tổng các lập phương của ba phần đó là 10728 . Hãy tìm số $\mathrm{M}$.
	\end{enumerate}
	\loigiai{} 
\end{bt}

\begin{bt}
	Cho tam giác $\mathrm{ABC}$ đều. Trên cạnh $\mathrm{AB}$ lấy điểm $\mathrm{D}$ sao cho $\mathrm{BD}=\frac{1}{3} \mathrm{AB}$. Tại $\mathrm{D}$ kẻ đường vuông góc với $\mathrm{AB}$ cắt cạnh $\mathrm{BC}$ tại $\mathrm{E}$. Tại $\mathrm{E}$ kẻ đường vuông góc với $\mathrm{BC}$ cắt $\mathrm{AC}$ tại $\mathrm{F}$.
	\begin{enumerate}[1.]
		\item Chứng minh $\mathrm{DF} \perp \mathrm{AC}$. Biết trong tam giác vuông cạnh đối diện với góc $30^{\circ}$ thì bằng nửa cạnh huyền.
		\item Chứng minh tam giác DEF đều.
		\item Gọi $G$ là trọng tâm của tam giác $D E F$. Chứng minh $G A=G B=G C$.
	\end{enumerate}
	\loigiai{}
\end{bt}

\begin{bt}
	Cho tam giác $\mathrm{ABC}$, trung tuyến $\mathrm{AM}$ và $\mathrm{BE}$ cắt nhau tại $\mathrm{G}$. Chứng minh rằng nếu $\mathrm{AGB} \leq 90^{\circ}$ thì $\mathrm{AC}+\mathrm{BC}>3 \mathrm{AB}$.
	\loigiai{} 
\end{bt}

\begin{bt}
	Tìm các giá trị nguyên của $x$ để biểu thức $C=\frac{22-3 x}{4-x}$ có giá trị lớn nhất.
	\loigiai{}
\end{bt}
