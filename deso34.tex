\onehalfspacing
\section{Đề số 34}

\begin{bt} 
	\hfill
	\begin{enumerate}[a.]
		\item Chứng minh: $5^{2014}-5^{2013}+5^{2012}$ chia hết cho 105 .
		\item Tìm số nguyên tố $\mathrm{p}$ sao cho $\mathrm{p}+2$ và $\mathrm{p}+4$ đều là số nguyên tố.
	\end{enumerate}
	\loigiai{} 
\end{bt}

\begin{bt}
	Tìm x biết:
	\begin{enumerate}[a.]
		\item $|3-2 x|=x+1$
		\item $\left(\frac{1}{2}+\frac{1}{3}+\ldots+\frac{1}{2014}\right) \cdot x=\frac{2013}{1}+\frac{2012}{2}+\ldots+\frac{2}{2012}+\frac{1}{2013}$
	\end{enumerate}
	\loigiai{} 
\end{bt}

\begin{bt}
	\hfill 
	\begin{enumerate}[a.]
		\item Tìm $x ; y ; z$ biết $\frac{x}{y}=\frac{3}{2} ; 5 x=7 z$ và $x-2 y+z=32$.
		\item Cho $\frac{7 x+5 y}{3 x-7 y}=\frac{7 z+5 t}{3 z-7 t}$. Chứng minh: $\frac{x}{y}=\frac{z}{t}$.
		\item Tìm giá trị nhỏ nhất của $\mathrm{A}=|x-2013|+|2014-x|+|x-2015|$.
	\end{enumerate}
	\loigiai{} 
\end{bt}

\begin{bt}
	Cho tam giác $A B C$ cân $(A B=A C)$. Trên cạnh $B C$ lấy điểm $D$ trên tia đối tia $C B$ lấy điểm $E$ sao cho $\mathrm{BD}=\mathrm{CE}$. Các đường thẳng vuông góc với $\mathrm{BC}$ kẻ từ $\mathrm{D}$ và $\mathrm{E}$ cắt $\mathrm{AB}$ và $\mathrm{AC}$ lân lượt ở $\mathrm{M}$ và N. Gọi I là giao điểm của $\mathrm{MN}$ và $\mathrm{BE}$.
	\begin{enumerate}[a.]
		\item Biết $\mathrm{AB}<\mathrm{BC}$. Chứng minh: $\hat{\mathrm{A}}>60^{\circ}$.
		\item Chứng $\operatorname{minh} \mathrm{IM}=\mathrm{IN}$
		\item Chứng minh đường thẳng vuông góc với $\mathrm{MN}$ tại I luôn đi qua 1 điểm cố định khi $\mathrm{D}$ thay đối trên cạnh $B C$.
	\end{enumerate}
	\loigiai{}
\end{bt}