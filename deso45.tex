\section{Đề số 45}

\begin{bt} 
	\hfill
	\begin{enumerate}[1.]
		\item Thực hiện phép tính: $A=\frac{\left(\frac{2}{5}\right)^7 \cdot 5^7+\left(\frac{9}{4}\right)^3:\left(\frac{3}{16}\right)^3}{2^7 \cdot 5^2+512}$.
		\item Cho $\frac{x+16}{9}=\frac{y-25}{16}=\frac{z+9}{25}$ và $2 x^3-1=15$. Tính $B=x+y+z$.
	\end{enumerate}
	\loigiai{} 
\end{bt}

\begin{bt}
	\hfill
	\begin{enumerate}[1.]
		\item Tìm $x, y$ biết: $x(x-y)=\frac{3}{10}$ và $y(x-y)=-\frac{3}{50}$.
		\item Tìm $x$ biết: $(x-3)\left(x+\frac{1}{2}\right)>0$.
	\end{enumerate}
	\loigiai{} 
\end{bt}

\begin{bt}
	\hfill
	\begin{enumerate}[1.]
		\item Tìm số tự nhiên n để phân số $\frac{7 n-8}{2 n-3}$ có giá trị lớn nhất.
		\item Cho đa thức $p(x)=a x^3+b x^2+c x+d$ với $a, b, c, d$ là các hệ số nguyên. Biết rằng, $p(x)$ $\vdots$ $ 5$ với mọi $x$ nguyên. Chứng minh rằng $a, b, c, d$ đều chia hết cho 5.
		\item Gọi $a, b, c$ là độ dài các cạnh của một tam giác. Chứng minh rằng:
		$$
		\frac{a}{b+c}+\frac{b}{c+a}+\frac{c}{a+b}<2 .
		$$
	\end{enumerate}
	
	\loigiai{} 
\end{bt}

\begin{bt}
	Cho tam giác $A B C$ cân tại $A$. Trên cạnh $B C$ lấy điểm $D(D$ khác $B, C)$. Trên tia đối của tia $C B$, lấy điểm $E$ sao cho $C E=B D$. Đường vuông góc với $B C$ kẻ từ $D$ cắt $A B$ tại $M$. Đường vuông góc với $B C$ kẻ từ $E$ cắt đường thẳng $A C$ tại $N, M N$ cắt $B C$ tại $I$.
	\begin{enumerate}[a.]
		\item Chứng minh $D M=E N$.
		\item Chứng $\operatorname{minh} I M=I N, B C<M N$.
		\item Gọi $O$ là giao của đường phân giác góc $A$ và đường thẳng vuông góc với $M N$ tại $I$. Chứng minh rằng $\triangle B M O=\triangle C N O$. Từ đó suy ra điểm $O$ cố định.
	\end{enumerate}
	\loigiai{}
\end{bt}

\begin{bt}
	Cho các số thực dương $a$ và $b$ thỏa mãn: $a^{100}+b^{100}=a^{101}+b^{101}=a^{102}+b^{102}$ \\Hãy tính giá trị của biểu thức: $P=a^{2014}+b^{2015}$.
	\loigiai{}
\end{bt}
