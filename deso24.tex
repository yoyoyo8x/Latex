\onehalfspacing
\section{Đề số 24}

\begin{bt} 
    Tính hợp lí:
   \begin{enumerate}[a.]
    \item $\frac{7}{-25}+\frac{-18}{25}+\frac{4}{23}+\frac{5}{7}+\frac{19}{23}$
    \item $\frac{7}{19} \cdot \frac{8}{11}+\frac{7}{19} \cdot \frac{3}{11}+\frac{12}{19}$
    \item $(-25) \cdot 125 \cdot 4 \cdot(-8) \cdot(-17)$ 
    \item $\frac{7}{35} \cdot \frac{10}{19}+\frac{7}{35} \cdot \frac{9}{19}-\frac{2}{35}$
   \end{enumerate}
\loigiai{}
\end{bt}

\begin{bt}
    Tính giá trị các biểu thức sau:
	\begin{enumerate}[a.]
        \item $A=\frac{1}{2}\left(1+\frac{1}{1.3}\right)\left(1+\frac{1}{2.4}\right)\left(1+\frac{1}{3.5}\right) ... \left(1+\frac{1}{2015.2017}\right)$.
        \item $B=2 x^2-3 x+5$ với $|x|=\frac{1}{2}$.
        \item $C=2 x-2 y+13 x^3 y^2(x-y)+15\left(y^2 x-x^2 y\right)+\left(\frac{2015}{2016}\right)^0$, biết $x-y=0$.
    \end{enumerate}
	\loigiai{} 
\end{bt}

\begin{bt}
    \hfill
    \begin{enumerate}[a.]
        \item Tìm $x, y$ biết: $\left(2 x-\frac{1}{6}\right)^2+|3 y+12| \leq 0$.
        \item Tìm $x, y, z$ biết: $\frac{3 x-2 y}{4}=\frac{2 z-4 x}{3}=\frac{4 y-3 z}{2}$ và $x+y+z=18$.
    \end{enumerate}
	\loigiai{}
\end{bt}

\begin{bt}
    \hfill
    \begin{enumerate}[a.]
        \item Tìm các số nguyên $x, y$ biết: $x-2 x y+y-3=0$.
        \item Cho đa thức $\mathrm{f}(x)=x^{10}-101 x^9+101 x^8-101 x^7+\ldots-101 x+101$. Tính $\mathrm{f}(100)$.
    \end{enumerate}
	\loigiai{}
\end{bt}

\begin{bt}
    Cho tam giác $\mathrm{ABC}$ có ba góc nhọn $(\mathrm{AB}<\mathrm{AC})$. Vẽ về phía ngoài tam giác $\mathrm{ABC}$ các tam giác đều $A B D$ và $A C E$. Gọi $I$ là giao của $C D$ và $B E, K$ là giao của $A B$ và $D C$.
    \begin{enumerate}[a.]
        \item Chứng minh rằng: $\triangle \mathrm{ADC}=\triangle \mathrm{ABE}$.
        \item Chứng minh rằng: $\widehat{\mathrm{DIB}}=60^{\circ}$.
        \item Gọi $\mathrm{M}$ và $\mathrm{N}$ lần lượt là trung điểm của $\mathrm{CD}$ và $\mathrm{BE}$. Chứng minh rằng $\triangle \mathrm{AMN}$ đều.
        \item Chứng minh rằng $IA$ là phân giác của góc $DIE$.
    \end{enumerate}
\loigiai{}
\end{bt}

\begin{bt}
    Cho tam giác $\mathrm{ABC}$ cân tại $\mathrm{A}, A=80^{\circ}$. Ở miền trong tam giác lấy điểm $\mathrm{I}$ sao cho $I B C=10^{\circ}, I C B=30^{\circ}$. Tính $A I B$
\loigiai{}
\end{bt}

