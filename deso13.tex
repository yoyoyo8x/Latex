\onehalfspacing
\section{Đề số 13}

\begin{bt} 
    \hfill
	\begin{enumerate}[a.]
		\item $\left|x+\frac{1}{5}\right|-4=-2$
        \item $2 x-\frac{1}{5}=\frac{6}{5} x-\frac{1}{2}$
        \item $(x-3)^{x+2}-(x-3)^{x+8}=0$
	\end{enumerate}
	\loigiai{} 
\end{bt}

\begin{bt}
	Tìm $x, y, z$ biết $\frac{x}{2}=\frac{y}{3}=\frac{z}{4}$ và $x^2+y^2+z^2=116$
	\loigiai{} 
\end{bt}

\begin{bt}
	Trong vòng bán kết giải bóng đá của trường THCS Phù Đổng có 4 đội thi đấu, gọi $\mathrm{A}$ là tập hợp các cầu thủ; B là tập hợp các số áo thi đấu. Quy tắc mỗi cầu thủ ứng với số áo của họ có phải là một hàm số không? Vì sao?
	\loigiai{}
\end{bt}

\begin{bt}
    Tính giá trị của đa thức $\mathrm{P}=x^3+x^2 y-2 x^2-x y-y^2+3 y+x+2017$ với
    $$
    x+y=2
    $$
\loigiai{}
\end{bt}

\begin{bt}
    Cho : $\frac{3 x-2 y}{4}=\frac{2 z-4 x}{3}=\frac{4 y-3 z}{2}$. Chứng minh: $\frac{x}{2}=\frac{y}{3}=\frac{z}{4}$
\loigiai{}
\end{bt}

\begin{bt}
    Tìm các số tự nhiên $\mathrm{x}$, $y$ thỏa mãn: $2 \mathrm{x}^2+3 \mathrm{y}^2=77$  
\loigiai{}
\end{bt}

\begin{bt}
    Cho $\triangle \mathrm{ABC}$, tia phân giác của góc $\mathrm{A}$ cắt $\mathrm{BC}$ tại $\mathrm{D}$. Biết $\mathrm{ADB}=85^{\circ}$
    \begin{enumerate}
        \item Tính: $\mathrm{B}-\mathrm{C}$
        \item Tính các góc của $\triangle \mathrm{ABC}$ nếu $4 . B=5 . C$
    \end{enumerate}
\loigiai{}
\end{bt}

\begin{bt}
    Cho $\triangle \mathrm{ABC}$ có ba góc nhọn, trung tuyến $\mathrm{AM}$. Trên nửa mặt phẳng bờ $\mathrm{AB}$ chứa điểm $\mathrm{C}$, vẽ đoạn thẳng $\mathrm{AE}$ vuông góc và bằng $\mathrm{AB}$. Trên nửa mặt phẳng bờ $\mathrm{AC}$ chứa điểm $B$, vẽ đoạn thẳng $A D$ vuông góc và bằng $A C$.

\begin{enumerate}
    \item Chứng minh: $\mathrm{BD}=\mathrm{CE}$
    \item Trên tia đối của tia MA lấy $\mathrm{N}$ sao cho $\mathrm{MN}=\mathrm{MA}$. Chứng minh: $\triangle \mathrm{ADE}=\Delta \mathrm{CAN}$.
    \item Gọi I là giao điểm của $\mathrm{DE}$ và $\mathrm{AM}$. Chứng minh: $\frac{\mathrm{AD}^2+\mathrm{IE}^2}{\mathrm{DI}^2+\mathrm{AE}^2}=1$
\end{enumerate}
\loigiai{}
\end{bt}