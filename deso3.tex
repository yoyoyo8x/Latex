\onehalfspacing
\section{Đề số 3}

\begin{bt} 
    Cho $\mathrm{x}, \mathrm{y}, \mathrm{z}$ là các số khác 0 và $\mathrm{x}^2=\mathrm{yz}, \mathrm{y}^2=\mathrm{xz}, \mathrm{z}^2=\mathrm{xy}$.\linebreak[2]
    Chứng minh rằng: $\mathrm{x}=\mathrm{y}=\mathrm{z}$.
\loigiai{} 
\end{bt}

\begin{bt}
    \hfill
    \begin{enumerate}[a.]
        \item Tìm $x$ biết: $5^x+5^{x+2}=650$
        \item Tìm số hữu tỷ $x, y$ biết: $(3 x-33)^{2008}+|y-7|^{2009} \leq 0$
    $$
    \left|\mathrm{x}-\frac{1}{2}\right|+\frac{3}{4}=\left|-1,6+\frac{3}{5}\right|
    $$
    \end{enumerate}
\loigiai{} 
\end{bt}

\begin{bt}
    Cho hàm số : $\mathrm{f}(\mathrm{x})=\mathrm{a} .\mathrm{x}^2+\mathrm{b} . \mathrm{x}+\mathrm{c}$ với $\mathrm{a}, \mathrm{b}, \mathrm{c}, \mathrm{d} \in \mathrm{Z}$

    Biết $f(1) \vdots 3 ; f(0) \vdots 3 ; f(-1) \vdots 3$. Chứng minh rằng $\mathrm{a}, \mathrm{b}, \mathrm{c}$ đều chia hết cho 3
\loigiai{} 
\end{bt}

\begin{bt}
    Cho tam giác $\mathrm{ABC}, \mathrm{AD}$ là tia phân giác của góc $\mathrm{A}$ và $B>C$.
    \begin{enumerate}[a.]
    \item Chứng minh rằng $A D C-A D B=B-C$.
    \item Vẽ đường thẳng $\mathrm{AH}$ vuông góc $\mathrm{BC}$ tại $\mathrm{H}$. Tính $A D B$ và $H A D$ khi biết $B-C=40^{\circ}$
    \item Vẽ đường thẳng chứa tia phân giác ngoài của góc đỉnh $\mathrm{A}$, nó cắt đường thẳng $\mathrm{BC}$
    tại E. 
    
    Chứng minh rằng $A E B=H A D=\frac{B-C}{2}$
    \end{enumerate}
\loigiai{}
\end{bt}

\begin{bt}
    \hfill
    \begin{enumerate}[a.]
        \item  Cho $S=1-\frac{1}{2}+\frac{1}{3}-\frac{1}{4}+\ldots+\frac{1}{2011}-\frac{1}{2012}+\frac{1}{2013}$ và $P=\frac{1}{1007}+\frac{1}{1008}+\ldots+\frac{1}{2012}+\frac{1}{2013}$.
        
        Tính $(S-P)^{2013}$.
        \item Cho $\mathrm{A}=\frac{\sqrt{x}+1}{\sqrt{x}-3}$
        
        Tìm $\mathrm{x} \in \mathrm{Z}$ để $\mathrm{A}$ có giá trị là một số nguyên.
    \end{enumerate}
\loigiai{}
\end{bt}