\onehalfspacing
\section{Đề số 23}

\begin{bt} 
    Thực hiện phép tính:
   \begin{enumerate}[a.]
    \item $A=\frac{155-\frac{10}{7}-\frac{5}{11}+\frac{5}{23}}{403-\frac{26}{7}-\frac{13}{11}+\frac{13}{23}}+\frac{\frac{3}{5}+\frac{3}{13}-0,9}{\frac{7}{91}+0,2-\frac{3}{10}}$
    \item $B=\frac{2^{12} \cdot 3^5-4^6 \cdot 9^2}{\left(2^2 \cdot 3\right)^6+8^4 \cdot 3^5}+\frac{5^{10} \cdot 7^3-25^5 \cdot 49^2}{(125 \cdot 7)^3+5^9 \cdot 14^3}$
   \end{enumerate}
\loigiai{}
\end{bt}

\begin{bt}
    \hfill
	\begin{enumerate}[a.]
        \item Chứng minh rằng: $3^{n+2}-2^{n+2}+3^n-2^n$ chia hết cho 10 với mọi số nguyên dương $\mathrm{n}$.
        \item Tìm giá trị nhỏ nhất của biểu thức : $A=|2014-x|+|2015-x|+|2016-x|$
        \item Tìm x, y thuộc $\mathrm{Z}$ biết : $25-y^2=8(x-2015)^2$
    \end{enumerate}
	\loigiai{} 
\end{bt}

\begin{bt}
    \hfill
    \begin{enumerate}[a.]
        \item Cho $\frac{x+16}{9}=\frac{y-25}{-16}=\frac{z+49}{25}$ và $4 x^3-3=29$. Tính: $\mathrm{x}-2 \mathrm{y}+3 \mathrm{z}$
        \item Cho $f(x)=\mathrm{ax}^3+4 x\left(x^2-1\right)+8$ và $g(x)=\mathrm{x}^3+4 x(b x+1)+c-3$ trong đó $\mathrm{a}, \mathrm{b}$, $c$ là hằng số. 
        
        Xác định $a, b, c$ để $f(x)=g(x)$.
    \end{enumerate}
	\loigiai{}
\end{bt}

\begin{bt}
    Cho tam giác $\mathrm{ABC}$ có $(\mathrm{AB}<\mathrm{AC})$. Gọi $\mathrm{M}$ là trung điểm của $\mathrm{BC}$. Từ $\mathrm{M}$ kẻ đường thẳng vuông góc với tia phân giác của góc $\mathrm{BAC}$ tại $\mathrm{N}$, cắt tia $\mathrm{AB}$ tại $\mathrm{E}$ và cắt tia $\mathrm{AC}$ tại $F$. Chứng minh rằng :
    \begin{enumerate}[a.]
        \item $\mathrm{BE}=\mathrm{CF}$
        \item $A E=\frac{A B+A C}{2}$
    \end{enumerate}
\loigiai{}
\end{bt}

\begin{bt}
   Cho tam giác $\mathrm{ABC}$ có góc $\mathrm{B}$ bằng $45^{\circ}$, góc $\mathrm{C}$ bằng $120^{\circ}$. Trên tia đối của tia $CB$ lấy điểm $\mathrm{D}$ sao cho $\mathrm{CD}=2 \mathrm{CB}$. Tính góc $\mathrm{ADB}$.
\loigiai{}
\end{bt}

