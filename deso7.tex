\onehalfspacing
\section{Đề số 7}

\begin{bt} 
	\hfill
	\begin{enumerate}[1.]
		\item Tính $M=\left(\frac{0,4-\frac{2}{9}+\frac{2}{11}}{1,4-\frac{7}{9}+\frac{7}{11}}-\frac{\frac{1}{3}-0,25+\frac{1}{5}}{1 \frac{1}{6}-0,875+0,7}\right): \frac{2012}{2013}$
		\item Tìm $x$, biết: $\left|x^2+\right| x-1||=x^2+2$.
	\end{enumerate}
	\loigiai{} 
\end{bt}

\begin{bt}
	\hfill
	\begin{enumerate}[1.]
		\item Cho $a, b, c$ là ba số thực khác 0 , thoả mãn điều kiện:
		$$
		\frac{a+b-c}{c}=\frac{b+c-a}{a}=\frac{c+a-b}{b} \text {. }
		$$
		Hãy tính giá trị của biểu thức $B=\left(1+\frac{b}{a}\right)\left(1+\frac{a}{c}\right)\left(1+\frac{c}{b}\right)$.
		\item Ba lớp 7A, 7B, 7C cùng mua một số gói tăm từ thiện, lúc đầu số gói tăm dự định chia cho ba lớp tỉ lệ với 5:6:7 nhưng sau đó chia theo tỉ lệ 4:5:6 nên có một lớp nhận nhiều hơn dự định 4 gói. Tính tổng số gói tăm mà ba lớp đã mua.
	\end{enumerate}
	\loigiai{} 
\end{bt}

\begin{bt}
	\hfill
	\begin{enumerate}[1.]
		\item Tìm giá trị nhỏ nhất của biểu thức $\mathrm{A}=|2 x-2|+|2 x-2013|$ với $x$ là số nguyên.
		\item Tìm nghiệm nguyên dương của phương trình $x+y+z=x y z$.
	\end{enumerate}
	\loigiai{} 
\end{bt}

\begin{bt}
	Cho $x A y=60^{\circ}$ có tia phân giác $\mathrm{Az}$. Từ điểm $\mathrm{B}$ trên $\mathrm{Ax}$ kẻ $\mathrm{BH}$ vuông góc với $\mathrm{Ay}$ tại $\mathrm{H}$, kẻ $\mathrm{BK}$ vuông góc với $\mathrm{Az}$ và $\mathrm{Bt}$ song song với $\mathrm{Ay}, \mathrm{Bt}$ cắt $\mathrm{Az}$ tại $\mathrm{C}$. Từ $\mathrm{C}$ kẻ $\mathrm{CM}$ vuông góc với Ay tại $M$. Chứng minh :
	\begin{enumerate}[a.]
		\item K là trung điểm của $A C$.
		\item $\triangle \mathrm{KMC}$ là tam giác đều.
		\item Cho $B K=2 \mathrm{~cm}$. Tính các cạnh $\triangle \mathrm{AKM}$.
	\end{enumerate}
	\loigiai{}
\end{bt}

\begin{bt}
	Cho ba số dương $0 \leq \mathrm{a} \leq \mathrm{b} \leq \mathrm{c} \leq 1$ chứng minh rằng: $\frac{a}{b c+1}+\frac{b}{a c+1}+\frac{c}{a b+1} \leq 2$
\end{bt}