\onehalfspacing
\section{Đề số 12}
\graphicspath{{./img/}}
\begin{bt} 
    \hfill
	\begin{enumerate}[a.]
		\item Tìm $x$ biết: $\quad \frac{1}{2016}: 2015 x=-\frac{1}{2015}$.
        \item Tìm các giá trị nguyên của $n$ để phân số $\mathrm{M}=\frac{3 n-1}{n-1}$ có giá trị là số nguyên.
        \item Tính giá trị của biểu thức: $N=x y^2 z^3+x^2 y^3 z^4+x^3 y^4 z^5+\ldots+x^{2014} y^{2015} z^{2016}$ tại: $\mathrm{x}=-1 ; \mathrm{y}=-1 ; \mathrm{z}=-1$
	\end{enumerate}
	\loigiai{} 
\end{bt}

\begin{bt}
	\hfill
	\begin{enumerate}[a.]
		\item Cho dãy tỉ số bằng nhau $\frac{2 b z-3 c y}{a}=\frac{3 c x-a z}{2 b}=\frac{a y-2 b x}{3 c}$. Chứng minh: $\frac{x}{a}=\frac{y}{2 b}=\frac{z}{3 c}$
        \item Tìm tất cả các số tự nhiên $m, n$ sao cho : $2^m+2015=|n-2016|+n-2016$.
	\end{enumerate}
	\loigiai{} 
\end{bt}

\begin{bt}
	\hfill
	\begin{enumerate}[a.]
		\item Tìm giá trị nhỏ nhất của biểu thức $\mathrm{P}=|x-2015|+|x-2016|+|x-2017|$.
        \item Cho bốn số nguyên dương khác nhau thỏa mãn tổng của hai số bất kì chia hết cho 2 và tổng của ba số bất kì chia hết cho 3 . Tính giá trị nhỏ nhất của tổng bốn số này ?
	\end{enumerate}
	\loigiai{}
\end{bt}

\begin{bt}
    Cho tam giác $\mathrm{ABC}$ cân tại $\mathrm{A}, \mathrm{BH}$ vuông góc $\mathrm{AC}$ tại $\mathrm{H}$. Trên cạnh $\mathrm{BC}$ lấy điểm $M$ bất kì ( khác $B$ và $C$ ). Gọi $D, E, F$ là chân đường vuông góc hạ từ $M$ đến $A B, A C$, $\mathrm{BH}$.
    \begin{enumerate}
        \item Chứng $\operatorname{minh} \triangle \mathrm{DBM}=\triangle \mathrm{FMB}$.
        \item Chứng minh khi $\mathrm{M}$ chạy trên cạnh $\mathrm{BC}$ thì tổng $\mathrm{MD}+\mathrm{ME}$ có giá trị không đổi.
        \item Trên tia đối của tia $C A$ lấy điểm $\mathrm{K}$ sao cho $\mathrm{CK}=\mathrm{EH}$. Chứng minh $\mathrm{BC}$ đi qua trung điểm của DK.
    \end{enumerate}
\loigiai{}
\end{bt}

\begin{bt}
   Có sáu túi lân lượt chứa $18,19,21,23,25$ và $34$ bóng. Một túi chỉ chứa bóng đỏ trong khi năm túi kia chỉ chứa bóng xanh. Bạn Toán lấy ba túi, bạn Học lấy hai túi. Túi còn lại chứa bóng đỏ. Biết lúc này bạn Toán có số bóng xanh gấp đôi số bóng xanh của bạn Học. Tìm số bóng đỏ trong túi còn lại.
\loigiai{}
\end{bt}