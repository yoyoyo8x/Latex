\onehalfspacing
\section{Đề số 21}

\begin{bt} 
    \hfill
    \begin{enumerate}[a.]
        \item Tính giá trị biểu thức $\quad \mathrm{A}=\left(2 \frac{1}{3}+3,5\right):\left(-4 \frac{1}{6}+2 \frac{1}{7}\right)+7,5$
        \item Rút gọn biểu thức $\quad B=\frac{2 \cdot 8^4 \cdot 27^2+4 \cdot 6^9}{2^7 \cdot 6^7+2^7 \cdot 40 \cdot 9^4}$
        \item Tính đa thức $\mathrm{M}$ biết rằng : $M+\left(5 x^2-2 x y\right)=6 x^2+9 x y-y^2$. Tính giá trị của $M$ khi $x, y$ thỏa mãn $(2 x-5)^{2018}+(3 y+4)^{2020} \leq 0$.
    \end{enumerate}
\loigiai{}
\end{bt}

\begin{bt}
    Tìm x biết: 
	\begin{enumerate}[a.]
        \item $-\frac{15}{12} x+\frac{3}{7}=\frac{6}{5} x-\frac{1}{2}$
        \item $\frac{1}{1.3}+\frac{1}{3.5}+\frac{1}{5.7}+\ldots .+\frac{1}{(2 x-1)(2 x+1)}=\frac{49}{99}$
        \item Tìm $x, y$ nguyên biết $2 x y-x-y=2$
    \end{enumerate}
	\loigiai{} 
\end{bt}

\begin{bt}
    \hfill
	\begin{enumerate}[a.]
        \item Tìm hai số nguyên dương $x$ và $y$ biết rằng tổng, hiệu và tích của chúng lần lượt tỉ lệ nghịch với $35 ; 210 ; 12$.
        \item Cho $$\frac{x}{y+z+t}=\frac{y}{z+t+x}=\frac{z}{t+x+y}=\frac{t}{x+y+z}$$. Chứng minh biểu thức $P=\frac{x+y}{z+t}+\frac{y+z}{t+x}+\frac{z+t}{x+y}+\frac{t+x}{y+z}$ có giá trị nguyên.
        \item Cho $\mathrm{a}, \mathrm{b}, \mathrm{c}, \mathrm{d} \in Z$ thỏa mãn $a^3+b^3=2\left(c^3-8 \mathrm{~d}^3\right)$.Chứng minh $\mathrm{a}+\mathrm{b}+\mathrm{c}+\mathrm{d}$ chia hết cho 3
    \end{enumerate}
	\loigiai{}
\end{bt}

\begin{bt}
    Cho tam giác $\mathrm{ABC}, \mathrm{M}$ là trung điểm của $\mathrm{BC}$. Trên tia đối của của tia $\mathrm{MA}$ lấy điểm $\mathrm{E}$ sao cho $\mathrm{ME}=\mathrm{MA}$. Chứng minh rằng:
    \begin{enumerate}[a.]
        \item $\mathrm{AC}=\mathrm{EB}$ và $\mathrm{AC} / / \mathrm{BE}$
        \item Gọi $I$ là một điểm trên $\mathrm{AC} ; \mathrm{K}$ là một điểm trên $\mathrm{EB}$ sao cho $\mathrm{AI}=\mathrm{EK}$. Chứng minh ba điểm $\mathrm{I}, \mathrm{M}, \mathrm{K}$ thẳng hàng
        \item Từ $\mathrm{E}$ kẻ $E H \perp B C(H \in B C)$. Biết $HBE=50^{\circ} ; MEB=25^{\circ}$.
        Tính $HEM$ và $BME$
    \end{enumerate}
\loigiai{}
\end{bt}

\begin{bt}
    Cho $B=\frac{3}{4}+\frac{8}{9}+\frac{15}{16}+\frac{24}{25}+\ldots+\frac{2499}{2500}$. Chứng tỏ $B$ không phải là số nguyên.
\loigiai{}
\end{bt}

